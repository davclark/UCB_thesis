\graphicspath{{experimental-description/}}

\chapter{Experimental Plan}

\section{Completed and in-process experiemnts}

A summary of completed and in-process experiments is given in
Table~\ref{table:collected}. Further explanation is given, where warranted, in
the text below.

\begin{table}
\begin{tabular}{lp{0.25\textwidth}p{0.5\textwidth}}
Paradigm & Description & Status \\ 
\hline \hline
Mechanism study 
    & UC Berkeley class (2010) & Fully analyzed, chap.~\ref{chap:ccc}. It
    remains to establish reliability in our coding scheme (see text). \\
    & UT Brownsville (2010) & Partially analyzed, no effect on climate attitudes \\
    & UT Brownsville (2011) & Data collected, but knowledge questions as yet
        uncoded, will be analyzed as with UC data \\

NDI study 
    & 2 UC Berkeley classes, 1 UT Brownsville (2011) & Data collected, currently
        being transcribed \\
\hline
\end{tabular}
\caption{Status of collected data-sets}
\label{table:collected}
\end{table}


\subsection{Mechanism Study}

While the methods are largely worked out for the mechanism study, there remains
a difficulty with coding the textual answers. If I am unable to develop an
efficient method for coding (i.e., automated, or otherwise train folks to code
more quickly), I will simply code a randomly chosen subsample of the data and
assume that the effects hold in the larger population (which will be used only
for attitude ratings, surprise, etc.) In any case, coding represents the only
remaining methodological piece of analyzing the mechanism data that hasn't been
finalized (although see section~\ref{sec:interviews} below for some
modifications to the common survey sections.

Currently, I have attempted a (relatively na\"ive) application of the na\"ive
Bayes algorithm to replicate codes (essentially a keyword-based approach which
ignores co-occurrences). This approach did not fare very well at all. I plan to
try similar methods to see if we are able to simply replicate the total scores.
In addition, Prof. Canny has recommended an exploration of "kernel based"
methods, in particular I plan to evaluate Latent Semantic Analysis and Latent
Dirichlet Analysis. In addition, Prof. Canny is offering a course in automated
text processing next semester.  This will be explored as time permits.

\subsection{NDI for climate change knowledge and attitudes}

A central limitation of the experiments in chapter~\ref{chap:two} was the lack
of a coherent message. Specifically, an intractable set of attitudes
would likely be affected by such a diverse collection of surprising facts,
including not only attitudes relevant to criminal justice, immigration, etc.,
but also to participants' sense of self-efficacy in terms of their numerical
reasoning!

The reasoning group has already developed an experiment-ready list of pro- and
anti-climate change acceptance numbers, and a similar list of evolution oriented
items is in process. These items will be used in experiments broadly similar to
that presented in chapter~\ref{chap:two}, but with the addition of an attitude
and belief survey as utilized in chapter~\ref{chap:ccc}.

This experiment will allow us to observe the effects of a collection of facts
and consequent psychological processing on a related belief (e.g., reality of
climate change) and attitudes (e.g., worry about climate change, intention to
change behavior).

In particular, I am excited to see how surprise regarding numerical information
compares to surprise in the mechanism study.

\subsubsection{NDI with evolution numbers}

Time permitting, a set of evolution-relevant numbers will be provided to
participants in addition to climate items. The RTMD theory suggests that by
engaging this related attitude, we may alter cognition with respect to climate
change (in particular learning and attitudinal evaluation and change). This is
not, however, a central aim of the thesis.

\section{List of proposed experiments}

\begin{table}
\begin{tabular}{llp{0.5\textwidth}}
Paradigm & Description & Plan \\
\hline \hline
On-line surveys
    & Mechanism &  Analogous to experiment in chap.~\ref{chap:ccc}. \\
    & Brief mechanism & If long-term attitude or conceptual change are observed
        using the full mechanism ``blurb,'' we may evaluate a shorter version,
        comprising only the shorter summaries from the blurb. \\
    & NDI (anti) & Directly analogous to the intervention used with Berkeley and
        Brownsville students this year. \\
    & NDI (pro) & Identical to the above, but with pro-climate change acceptance
        numbers. \\
    & NDI (evo) & An intervention utilizing a mixture of pro-climate and
        pro-global warming numbers. This may enhance the effect of the
        intervention due to something like general pro-science sentiment. \\

Clinical interviews 
    & Mechanism & Prior to aggressive nation-wide data collection,
    semi-structured interviews will be conducted both during and after the
    on-line version of the survey w/ RPP students. If warranted, this may be
    attempted with a laptop with the general public in public place, etc. \\
    & NDI & Similar to the above, focused on pro- and anti-climate change
        statistics \\
\hline
\end{tabular}
\caption{Proposed extensions of the above paradigms}
\label{table:proposed}
\end{table}

\subsection{On-line survey methodology}

Surveys will be delivered using qualtrics. Qualtrics provides for
straightforward subject tracking, which will allow us to administer follow-up
surveys with relative ease. CPHS approval has already been approved for these
approaches, including the ability to follow up with experimental participants
for a period of up to one year. This straightforward extension of our previous
interventions will allow us to answer one of the central questions in evaluating
their utility: ``Do our interventions have any long-term impact on knowledge
and/or attitudes?'' Both simpler regression-style approaches as well as full
blown growth models (a form of structural equation modelling) will be used to
evaluate these changes. Given practical concerns, while I plan to collect
longitudinal data at longer intervals, followups collected at one week
post-survey will be used to inform the thesis.

\subsection{Expansion and refinement of survey items}

In the proposed studies, I plan to continue the use versions of the instruments
described above regarding climate change and RTMD. Specifically, I'll continue
to use a mechanistic explanation of global warming, as well as an RTMD attitude
survey. In addition to these, the reasoning group continues to refine lists of
both pro- and anti-global warming and -evolution NDI style items. Given the
thesis' focus on psychological processing, variation in the stimulus set may
serve to lend credibility to the generalizability of any findings. However, care
will be taken to ensure that different sets of items are not provided to
fundamentally different subject populations. As a final point of comparison,
Joseph Williams (a collaborator) is planning to compare our mechanistic
explanation to other forms of argumentation from e.g. Al Gore's Climate Project
and the EPA website on climate change.

A basic issue regarding our ability to apply factor-based models is that we
should have three or more items per construct. This is a trivial problem, and I
plan to include additional items from Callie's study to bolster numbers for
under-represented constructs that are central to our experiment going forwards.
This will allow a more statistically sound evaluation of shifts in both the mean
values of constructs as well as the correlations between them.

\subsubsection{Undergraduate Interviews}
\label{sec:interviews}

A few constructs are not currently assessed in a highly accurate fashion. While
I had initially been considering applying physiological methods for capturing,
e.g., emotional response to survey items and the information presented, I am
instead taking a cue from psychophysics. To wit, individuals are profoundly more
sensitive to their own internal states than even the best physiological data
collection and analysis methods (notably, there may be aspects of experience
that are inaccessible to experience, and this may warrant followups with more
physiological measures). Thus, in order to gain greater purchase on participants
surprise, attitudes towards climate change, metacognition on their own knowledge
and learning, and so on. In particular, I plan to include more focused questions
regarding participants' responses to the material, including prompts to separate
what information was simply ``new'' as compared to ``convincing,''
``surprising'' or ``emotionally impactful.''

Permission for such interviews from CPHS is a pre-requesite to engaging in this
approach. While waiting for approval of this modification, I will conduct
surveys with RPP without interviews in order to pilot the on-line survey.

\subsection{Longer-term evaluation of learning and attitude change}

A surprising non-finding is that surprise is not terribly predictive of anything
in the mechanism experiment. This is likely because we have not allowed time for
the immediate effects of learning to dissipate. I expect that simply by
extending the time between the intervention and the post-test (at various lags
from a day to at least a few weeks), we will begin to see significant
predictions based on surprise. Specifically, I'd expect surprise (and other
affective responses) to predict better retainment or ``consolidation'' of the
factual knowledge, as well as more durable shifts in beliefs and attitudes. I
would expect enhancement of episodic encoding to have more of an effect towards
the factual end of subjects internal representations. Surprise, on the other
hand is likely have have broad effects.

\subsection{Subject selection}

Subject selection is a problem of central importance, given the inhomogeneity in
the population surrounding climate change and evolution acceptance. While, for
practical reasons, experiments will generally be run first at UC Berkeley, we
have already begun to collect data from remote locations. Specifically, we have
already obtained participants from a climate physics class in a southern Texas
university, many of whom reject climate change. In addition to recruiting
students from diverse universities, participants may be recruited from public
locations (Malls, DMVs), as well as from on-line. Based on the recommendation of
Prof. Canny and Joseph Williams, Facebook and Craig's list are the most likely
target for on-line participant recruitment, although other mechanisms will be
considered.

For the purposes of this thesis, I will seek to replicate the most central
results in a broader population as described above. In practical / contractual
terms, I expect to collect a total of 3 such experiments, distributed between
on-line and "real-life" populations.

Some of the potential effects we may observe are embedded within a large number
of variables, and may be relatively small (we cannot expect a lifetime of
climate change-related experience to be eradicated by a 10-minute
intervention!). Some require data-heavy techniques, like SEM to do regression on
latent variables. Thus, a primary experimental thrust will be to engage several
hundred participants in an on-line, somewhat expanded version of the experiment
described in chapter~\ref{chap:ccc}.
