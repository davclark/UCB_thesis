\graphicspath{{evil-ndi/}}

\chapter{Learning \texorpdfstring{``Evil''}{"Evil"} Non-Representative Climate Numbers}
\label{chap:evilndi}

The Numerically Driven Inferencing (NDI) is introduced in
Chapter~\ref{chap:intro}, Section~\ref{sec:ndi}. In Chapter~\ref{chap:two}, we
see that surprising numerical information can have a lasting impact on an
individual's conception of a politically relevant number--sometimes even without
conscious recollection of the experience that catalyzed this change.  Below, we
will see that misleading, cherry-picked numerical facts can have a marked shift
on individuals' beliefs and policy preferences. This result serves as a
call-to-arms for climate educators, as even relatively well-educated, liberal,
GW-accepting students are highly susceptible to the kinds of information
currently being used to undermine belief in climate change.

\section{Overview}

As described in Chapter~\ref{chap:intro}, Section~\ref{sec:comm-strategies},
there is some debate surrounding the value of scientific or numeric information
regarding climate change. Indeed, some have claimed that such interventions will
only serve to polarize individuals, thus making the situation worse than it
already is. However, some organizations publish out-of-context facts to try to
undercut the reality or gravity of human-caused climate change. Such numbers are
often blatantly cherry picked. For example, one con locate GW deniers who note
that the Earth’s temperature decreased (by 0.2°F) from 1940 to 1975
\parencite{jastrow_global_1991}. This surprising fact, though, hardly
contradicts the ever more obvious warming trend over the last 125+ years---one
can pluck many ``trends'' in noisy time series by picking endpoints that are
oddly high or low. Given this rather clear intent to mislead
\parencite[corroborated by][]{oreskes_merchants_2010}, we (partly
tongue-in-cheek) label these numbers ``evil.'' 

Our three hypotheses were that misleading facts would reduce:
\begin{enumerate}
    \item participants’ climate change acceptance, 
    \item ratings of their knowledge of the issue, and 
    \item their climate-change funding preferences.
\end{enumerate}
Of course, lest we would have eroded participants’ acceptance of anthropogenic
climate change more than fleetingly, we debriefed them right afterward with more
complete information---including a large dose of representative numerical facts
as well as the basic physical mechanism of GW (the mechanism is detailed in full
in Chapter~\ref{chap:mechanism}). It should be clear that we're not interested
in perfecting an approach to eroding acceptance of the overwhelming scientific
consensus on climate change! 

In this Chapter, we'll see easily obtained and dramatic effects with an
anti-climate NDI intervention. There is only one experiment reported, with the
intent of determining whether these materials would drive meaningful
changes. In particular, we administered this experiment only to undergraduate
seminar courses where we could be sure to administer a proper debriefing.

\section{Study: UC classroom intervention with \texorpdfstring{“evil”}{"evil"}
    numbers}

\subsection{Methods} 
\label{sec:evilndi-methods}

\subsubsection{Materials and Procedure}

Participants were engaged in one of two similar misleading numeracy
interventions. In both versions, survey methods were as described in
Chapter~\ref{chap:survey}. This study utilized a somewhat compact version of a
pre- and post-intervention test using only the 14 items in
Table~\ref{table:rtmd-questions} (up through \textsf{engage}), plus a
self-rating of climate-change knowledge.  

In the “no pre-test blast” version of the intervention, participants estimated
each of eight items prior to receiving the feedback values, with an emphasis on
maximizing the quantity of feedback numbers presented to the participant. To
this end, this eight-item survey included only a post-test (i.e., no pre-test),
and lacked a policy component (thus, it was an EI intervention, lacking ``P'' or
``C'', simlar to the approach used in Chapter~\ref{chap:two}). 

A more comprehensive engagement containing only two items was administered to
the rest of the class. This version included a pre-test and additional questions
about each item. In addition, we asked students about their surprise level after
each feedback value and requested both their climate-change funding Policies and
post-feedback policy Changes versus various UNDP millennium goals.  Thus, this
latter variant was a full EPIC intervention. The same set of alternatives was
used across the 4 variants of the 2-item intervention, and these are listed
along with policy-relevant instructions in Appendix~\ref{app:undp}.

Note that while this experiment is presented first as a motivation for the
following chapters, it was actually carried out \emph{after} a number of
experiments in Chapters~\ref{chap:prondi} and \ref{chap:mechanism}. Thus, a
number of experimental design choices made here are motivated by findings from
experiments in those chapters.

\subsubsection{Participants}

Two classes of UC Berkeley undergraduates (spanning cognitive science and
“Behavioral Change”) were engaged in this intervention ($N=104$). 59 students
completed the 8-item “no pre-test” version of the experiment and 45 completed
the 2-item full EPIC intervention. All participants were retained after
examining the coherence of survey responses.

In the 8-item intervention, 34 participants were female and mean conservativism
of 3.64 ($sd=1.59$). In the 2-item intervention, 31 participants were female and
mean conservativism was 3.68 ($sd=1.43$). Breakdown of political party is given
in Table~\ref{table:evil-party}.

% TODO: party table 
% latex table generated in R 2.15.1 by xtable 1.7-1 package
% Tue Jun 18 14:46:50 2013
\begin{table}[ht]
\caption{Stated party affiliations for participants in “evil” NDI study (UC
    Berkeley undergraduates).}
\label{table:evil-party}
\centering
\begin{tabular}{rrr}
  \toprule
     & 8-item & 2-item \\ 
  \midrule
  democrat &  20 &  21 \\ 
  republican &   4 &   2 \\ 
  green &   1 &   0 \\ 
  libertarian &   1 &   3 \\ 
  independent &   6 &   2 \\ 
  none &  21 &  13 \\ 
  other &   1 &   1 \\ 
  decline to state &   5 &   3 \\ 
   \bottomrule
\end{tabular}
\end{table}
\subsection{Results}

% TODO: Also note changes for each item / individual comparisons here (if you
% have time / people want)
Overall, these numbers had a profound impact, the details of which are described
below. As with other NDI interventions, and consistent with our pilot testing,
individuals generally found each of these items surprising, ranging from
surprise ratings of 5.83 to 8.53 across both interventions. Mean surprise
ratings were 6.03 for the 2-item intervention and 6.62 for the 8-item
intervention. Ratings were on a 1--9 scale, with all ratings above “1”
indicating some level of surprise.

\subsubsection{Shifts away from GW policy preferences}

As hypothesized, policy preferences for funding UN goals related to climate
change dropped ($\chi^2(1)=22$, $p<0.01$) for all eight funding priorities.
(Unfortunately for global warming as a social priority, the highest mean
pre-test preference for funding climate change initiatives reached only a 50-50
split of available funds.) These results are depicted in
Figure~\ref{fig:evil-alloc}. While, due to time constraints, we did not check
for a similar result in our 8-item intervention, it seems likely that similar
(or greater) shifts would occur along with the much more drastic GW attitude
shifts we'll see below.

\begin{figure}
    \centering
    \includegraphics{evil-alloc.pdf}
    \caption{Significant drops in preference for allocation of \$100M towards
        climate-related projects on “UN Advisor” task ($p<5\times10^{-6}$). Each
        of the 4 survey variants is represented by a different color. The two
        policy choices remained the same (indicated by solid vs. dashed lines),
        while numerical information was varied across surveys. Values are
        expressed as a percentage of funds that were allocated to
        climate-relevant projects vs. projects supporting an alternative UNDP
        millenium goal.}
    \label{fig:evil-alloc}
\end{figure}

\subsubsection{GW acceptance eroded by misleading numbers}

Also, as hypothesized, mean climate change acceptance dropped significantly,
from 6.5 on the pre-test to 6.2 on the post-test for the two-item group (6\% of
available room, for a 9-point scale, $t(42)=-4.3$, $p<0.001$), and significantly to
5.9 for the eight-item group (12\% of available room, $t(88.6)=‑2.61$, $p<0.005$).
Note that these shifts were also in the direction of ambivalence (a ``5''
rating), and may reflect confusion rather than disagreement. Mean ratings are
depicted in Figure~\ref{fig:evil-GW}.

\begin{figure}
    \centering
    \includegraphics{evil-GW.pdf}
    \caption{Mean ratings for GW survey items on pre- and post-test. Pre-test
        surveys were only administered for the 2-item group, but should be
        indicative of population responses.}
    \label{fig:evil-GW}
\end{figure}

\subsubsection{Self-confidence in GW knowledge eroded by misleading numbers}

Our third hypothesis was also supported, as self-rated knowledge dropped from a
mean of 5.0 on the pre-test to 4.5 for the two-item group (12\% of available
room, $t(44)=-2.5$, $p<0.01$), and plummeted to 2.9 on eight-item survey
($t(87.2)=-5.3$, $p<0.001$). This latter decrease, 2.1, represents 53\% of the
available room to drop on a 9-point scale, which is exceptionally large. These
ratings are depicted in Figure~\ref{fig:evil-know}.

\begin{figure}
    \centering
    \includegraphics{evil-know.pdf}
    \caption{Self-rated knoweldge for individuals on pre- and post-tests. Again,
        pre-tests were only administered to individuals in the 2-item
        experimental variant.}
    \label{fig:evil-know}
\end{figure}

\subsection{Discussion}

In stark contrast to arguments that numeracy is polarizing
\parencite{kahan_polarizing_2012}, we have provided an existence proof that
appropriately selected scientific facts can have a profound effect in eroding
the existing beliefs of a population (i.e., “liberals” can be pushed in a more
“conservative” direction). In particular, we have demonstrated marked erosion of
self-confidence in one's own knowledge, as well as belief and concern regarding
anthropogenic climate change---even in our relatively liberal and
anthropogenic-climate-change-accepting sample of UC Berkeley undergraduates.
Such results were observed with as little as \emph{two} numbers.
% Still need to grab stuff from above
Consider the effect of Muller's writings prior to \textcite{rohde_new_2013}. A
prominent professor at our ostensibly liberal institution wrote extensively on
why we should doubt the veracity of climate change---including in the mainstream
media. We must assume that educated, liberal, GW-accepting individuals may be
easily swayed by a small dose of factual (but non-representative) numerical or
scientific information from such a source.

The primary point illustrated by this study is that individual's understandings
are demonstrably fragile. Even an intervention of a few minutes can massively
undercut individual's confidence in thier own knowledge, along with overall
belief and concern about global climate change. An additional point is that, as
noted by \textcite{kahan_polarizing_2012,mccright_politicization_2011}, surveyed
individuals were little-affected by self-guided educational efforts. Thus, one
might conclude that climate change accepters are unlikely to come into contact
with such numbers on their own.  However, there are concerted efforts to
distribute such numbers on the internet and elsewhere. A final point is that, as
shown above and as noted by \textcite{mccright_politicization_2011}, scientific
information might push individuals both towards scientific consensus, as well as
\emph{away} from it.  Thus, it seems wise to build a solid foundation of climate
change relevant knowledge in the American populace. 

It is clear that even relatively educated members of the public (e.g.,
undergraduates at a top-tier university) are highly susceptible to misleading,
cherry picked facts. Such facts are clearly known to organizations attempting to
undermine the overwhelming scientific consensus about climate change. Thus,
climate educators and communicators must counter the increasing sophistication
with which such organizations distribute misleading information.

\section*{Acknowledgements}

The work reported in this chapter has been previously published, in part, in
\textcite{clark_knowledge_inpress}.  All such material is re-used here with the
permission of my co-authors, the publishers, and the Graduate Division at the
University of California, Berkeley.
