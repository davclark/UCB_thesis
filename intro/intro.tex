\graphicspath{{intro/}}

\chapter{Introduction} \label{chap:intro}

Anthropogenic climate change is real, and it's a behavioral problem. Behavioral
change is challenging, and can proceed along any of a number of routes. An
increasingly vocal set of climate communications researchers argues that science
education is not effective, and some even argue that it is
\emph{counter-productive}.  However, as with many failures to observe a result
in the behavioral sciences, it may simply be that effective climate-related
science education, is simply very hard. Indeed, as we'll review below, there are
ready examples of the effectiveness of science education---both in terms of
knowledge gains and in terms of policy-relevant preferences. Thus, the
groundwork is well-laid for a careful consideration of science and numeracy
interventions targeting climate change.

\section{The Problem of Global Climate Change}

Scientific controversy has been a major element of public discourse for the last
half-century (for a comprehensive review, see \cite{oreskes_merchants_2010}).
Perhaps no issue sparks as much controversy today as global climate change
(commonly referred to as global warming, from which we'll derive the
abbreviation “GW”). As such, a psychological consideration of the problem of
climate change entails some basic background in the fundamental climate science,
and a dash of moral reasoning. In the chapters that follow, we'll see empirical
results supporting the utility of these basic science materials for the
development of successful interventions.

\subsection{We're All Going to Die}

Apart from those lucky few deathless microbes who might catch a ride out of this
solar system before the sun goes extinct, all organisms on earth today (or
decended from such) will ultimately find themselves dead. Even us. This is, in
my opinion, no great cause for alarm, though arguments to this effect fall
outside of the scope of this dissertation. (And so, on this point, I must refer
you to other sources of your own choosing.) I hope I needn't argue this basic
fact---a fact that is necessary for clarity in what we might (or should)
possibly hope to accomplish in mitigating climate change.

So, we cannot keep people from dying or species from going extinct. These are
certain eventualities. This may provide for some moral ambiguity, but for the
purposes of this dissertation, we might adopt what I hope will be a relatively
uncontroversial notion. Specifically, our actions today may lead to more or
less suffering for ourselves, and (as we'll see below) particularly for our
children and their decendants. Likewise, our actions stand to have a marked
impact on the potential richness of biological diversity on this planet.
Howsoever our moral compass may direct us in the face of such realities,
that compass cannot function without a basic orientation to the likely---or, in
some cases, certain---consequences of our actions.

\subsection{The Stark Reality of Climate Change}

% Maybe also some John D'Agata? Or perhaps in concluding chapter? Also military
% concerns?
% This part was partially adapted and clarified from the CogSci 2013 article
% intro

Many are familiar with the United Nations' International Panel on Climate Change
(IPCC). This organization produces a thorough summary of the current science,
though only every few years. Thus, for more up-to-date information, one can turn
to organizations such as the World Bank. We find, for example, that our
atmosphere's carbon dioxide (CO\textsubscript{2}) concentration is higher now
than in any of the past 15 million years \parencite{world_bank_turn_2012}.
Looking forward, there is now a 40\% chance that our world will be 4°C warmer by
2100. This warming would have harsh consequences for all people, but would
disproportionately affect the poorest individuals.  Some of these effects are
already present and require adaptation or preparedness
\parencite{potsdam_institute_for_climate_impact_research_and_climate_analytics_turn_2013}.
Global warming akin to recent and projected trends last occurred over 17 million
years ago, when a 3-4°C gain occurred over 1,500,000 years. Thus, the projected
timescale of 100 years is over 10,000 times faster than previous timescales
\parencite{barnosky_heatstroke:_2009}. In previous warming periods of this
magnitude, widespread extinctions occurred \parencite{mayhew_long-term_2008}. We
must therefore assume that we stand to lose a staggering number of species in
the timescale of our childrens' or grandchildrens' lifetimes. This would include
species comprising the biological systems we all depend upon (e.g., for food).

Given the timecourse of carbon dioxide and other greenhouse gases in the
atmosphere, the primary variable of interest will be total emissions until we
reach a carbon neutral economy. Plausible best-case and worst-case scenarios are
outlined by \textcite{archer_millennial_2008}, and these predictions are stark:
we have likely already “baked” a certain amount of warming into the climate
system that will last for at least 100,000's of years. And, if we do not
aggressively change our current behavior, these changes will be on the order of
6°C or more (which, I hope I needn't point out, will have effects worse than
those from 4°C increases detailed in the World Bank reports mentioned above).

It is also necessary to point out the fact that there has been vocal opposition
to the above arguments for action. Some of the more credible have attacked the
notion that climate action is appropriate on economic grounds
\parencite[e.g.,][]{lomborg_cool_2007}. Other scholars, like Richard Muller,
have claimed that the science is “uncertain” (although this view was
recanted in a report produced by his own research initiative
in \cite{rohde_new_2013}). 
% In a fuller exposition, might mention that this report ended up claiming
% warming was larger than previously asserted by the IPCC
While a full rebuttal of these arguments is beyond the
scope of this dissertation, it is also fortunately not necessary---citizen
scientists have created publicly available resources that do this already (most
notably the Skeptical Science website).  Such opposition to science by
entrenched interests is nothing new, and is well-chronicled by
\textcite{oreskes_merchants_2010}.

\section{Climate Change as a Behavioral Problem}

Nearly all climate researchers have concluded that this problem is urgent and
anthropogenic (i.e., essentially 100\% caused by human behavior). 
% Could include ref from e.g., the science daily article I sent to Michael in
% Spring 2013
As such, it will be ``solved'' only by changes in human behavior.
\textcite{harte_cool_2008} provide one such plan for action, detailing methods
to rapidly reduce emissions. \textcite{schrag_hope_2011} provides an alternative
perspective on what will be required, underscoring the importance of considering
total emissions before arriving at a carbon-neutral energy economy. While both
focus on emissions, these plans have a markedly different focus. Given such
examples, it is clear that a large part of the necessary behavior will be
discourse leading to an agreed and appropriate plan of action. Below, we'll see
some arguments that greater public acceptance and political will beyond
current levels will almost certainly be required to achieve followthrough on a
reasonable plan.

\subsection{Behavior is Embedded Within our Economic System}

\textcite{hoffman_climate_2010} suggests that while smoking provide a useful
(and recent) example of a successful shift in science-related policy and
attitudes, he goes on to suggest that slavery may provide a more apt model. A
central differentiating feature between these two examples is that slavery was
central to the ante-bellum economy, much as carbon-intensive energy production
is now.  One cannot simply substitute another vice, as one could with tobacco
consumption.  Rather, dramatic and deep changes must be made to the manner in
which our current economic system functions, including the way we produce
everything from food to luxury goods. Indeed, the availability of fossil fuels
and other energy sources was likely critical to the shift away from slavery.

It is unlikely that carbon-neutral policy will result in anything approaching
the conflict of the American civil war. It is clear, however, that there will
continue to be strong opposition to such policy from a wide variety of economic
sectors. Thus, from a policy point of view, broad public agreement stands to
play a key role in overcoming such opposition.

\subsection{Americans are Weird}

While humans continue to increase our understanding of the world, issues like
climate change are not readily comprehended by non-specialists. The IPCC
(Intergovernmental Panel on Climate Change) and Skeptical Science have assembled
and disseminated the scientific consensus on GW, but, sadly, the U.S.  public
remains divided on both the existence and the cause of climate change
\parencite[cf.][]{hoffman_growing_2011}. Indeed, while much of the developed
world accepts anthropogenic climate change as a reality, as of January 2010 only
57\% of individuals surveyed in the United States think global warming is
happening at all. When asked to assume that global warming \emph{is} happening,
only 47\% of the same group of respondants indicated that they thought it was
``caused mostly by human activities'' \parencite[Q47 and Q50
in][]{leiserowitz_climate_2010}.  Presumably, the number of Americans accepting
anthropogenic climate change is somewhat less than this figure. 

While some \parencite[e.g.,][]{krosnick_does_2013} argue that American's
acceptance and concern is somewhat higher than indicated above, it seems
indisputable that Americans are consistently outliers in this and many other
dimensions. Multiple sources illustrate that Americans lag almost all other
surveyed nations on GW acceptance.  \textcite{ray_worldwide_2010} report that
the United States lags all of the 110 other surveyed nations with 47\% of
respondents reporting that “rising temperatures are a result of natural causes”
(i.e., not the result of human causes). This sample included both peer nations
and developing countries.  \textcite{leiserowitz_international_2007} describes
how even at the height of American GW acceptance, we lagged much of the rest of
the world.

%% This seems not necessary, but could be added back if necessary

%% Canny thinks the TTM cited here is not terribly relevant
% Thus, if we want
% to do something about this issue in the context of a democratic society, the
% first step is getting a reasonable majority of people to accept that there at
% least \emph{may} be a problem \cite{prochaska_toward_1986}.

\textcite{ranney_why_2012} provides a more comprehensive overview of American
exceptionalism.  A general problem identified within Ranney's Reinforced
Theistic Manifest Destiny (RTMD) framework is the fact that individuals often
reject scientific ideas when they are in conflict with their other attitudes and
beliefs. It is as if we are endowed with something of a conceptual immune system,
% Maybe elaborate more
comprising religious and nationalistic beliefs in some, and more scientifically
grounded beliefs in others. When an individual is “exposed” to ideas in
opposition to their currently held beliefs, these beliefs serve as a defense
against any potential conceptual changes \parencite[as also discussed
by][]{shepherd_perpetuation_2012}.

Ranney mentions a number of similarities between evolution and climate change
related cognition, but there is a sharp distinction between the kinds of
``emotional responses'' that might be experienced in response to these
scientific ideas. Specifically, climate change is something that involves the
ethical status of actions that we do every day, both individually and as a
society.  Evolution, on the other hand, tends to incohere with personally held
religious beliefs, most directly divine creation, but then by extension, deity
and afterlife. Thus, Americans appear to require more convincing than citizens
of other nations when it comes to scientific assertions, but there are likely
nuances particular to each scientific domain---in particular, we should
understand precisely where conceptual conflicts may arise. So, we may find
some useful examples in successful evolution education interventions, but we must
be mindful that the two domains are quite distinct in some ways.
% The above glosses over "green dragon" notions in the evangelical community,
% but this is a bit out of scope

% Also note story of stuff re: american exceptionalism?

\section{Climate Communication Strategies}
\label{sec:comm-strategies}

% Insert here a more thorough description of the knowledge deficit camp, find
% stuff I sent to Michael as part of the CogSci, particularly Hansen and others
% who introduce the "knowledge deficit" terminology
A group of climate communication researchers, oddly, suggests that educational
ventures would be of little or no help. \textcite{kahan_polarizing_2012} found
that, for the U.S. (a high per-capita carbon user), numeracy and scienctific
literacy were correlated with more biased views on GW. Specifically,
“hierarchical individualists” (Kahan's term that approximates “conservatives”)
were more likely to reject the reality of GW if they ranked highly on science
and numerical literacy. Similarly, \textcite{mccright_politicization_2011}
highlight data indicating that the ``education level'' effect on climate belief
is moderated by conservatism/affiliation (with conservative or ``Republican'' GW
denial being slightly positively related, if at all, with education). This (also
correlational) evidence, they claim, disproves a naïve ``knowledge deficit''
view---that is, the view that more education can shift the public's beliefs
toward the scientific consensus about climate change. 

The above position harkens back to a classic social psychology report by
\textcite{lord_biased_1979}, which reported that people with a strong position
tended to polarize further after receiving information that was contrary to
their views (though this information was  not particularly factual).
Interestingly, research from our own lab has refuted the applicability of this
result to educational policy. The above study removed the middle third of
individuals from the population, leaving only those with relatively strong
views. However, \textcite{nelson_criminal_2007} shows that a persuasive
intervention will in fact have the expected net effect if the population as a
whole is considered. Certainly, a shift in the middle third of the United States
on the issue of climate change would be sufficient to amass the kind of
political will alluded to above.

We can even find hints for the development of successful science education
interventions in the observations of \textcite{mccright_politicization_2011}.
Specifically, this work indicates a bifurcation in the kinds of information that
liberals and conservatives tend to receive.  This split leaves open the
possibility that well-constructed interventions may indeed induce conservatives
to accept the scientific consensus (with little challenge to their core values).  

It cannot be emphasized enough that while overreaching one's results may be
close to required in todays academic publishing climate, such a practice borders
on irresponsible with respect to critical policy-relevant science. Certainly,
one should be cautious in stretching a result from supporting “science education
approaches are difficult” to “science education approaches \emph{don't work}”.
The results of such publications are difficult to quantify, but one can turn to
the mainstream press to see how such results are reported. In a recent article
in Mother Jones, \textcite{mooney_science_2011} provides a summary of much of
the above research, arguing against “the standard notion that the way to
persuade people is via evidence and argument.” I can also offer an anecdote: I
met with an individual working at a prominent science museum who was in the
process of managing a 4-city deployment of a climate-change exhibit.  Based on
her reading of the literature, they decided to err on the side of avoiding
polarization, and excluded scientific descriptions of anthropogenic climate
change. Thus, based on a reading of the current literature, \emph{science
    museums} are deciding to \emph{withhold science education} for fear of
polarizing the public.

This kind of problem is not limited to science education, but rather seems to be
part of a general trend. For example, \textcite{sachs_field_2012}, after evaluating
hard-to-use vs. easy-to-use programmable thermostats, report that their
provision of easier-to-use thermostats had no significant effect on heating
energy usage. Here, I must again resort to anecdote: this result was then
presented at a conference I attended as definitive evidence that “programmable
thermostats don't work.” 
% This is pretty disheartening, but more disheartening are results such as from
% Opower of 2% reductions being considered large!
This is akin to exploring two iterations in an
engineering design process, and then declaring that the problem is insoluble. We
would have never succeeded with the moon shot, or with mobile handheld computing
with such thinking!\footnote{The Apple Newton, for example, was released in
    1993---almost 10 years after the Psion Organizer was released in 1984. Even
    with Apple's legendary engineering capacity and existing commercial models
    to learn from, the Newton was a flop. Using such failures to argue that
    handheld computing “doesn't work” clearly leads to a logical contradiction
    with today's reality!}
As outlined above, climate-relevant behavior change is a hard problem in
general.  Thus, a successful intervention is likely going to require much
careful thought and iteration. In this document, we'll see an example of such a
process.

% One’s capacity as a philosopher need not be
% terribly advanced to infer that disbelief in a problem will likely inhibit any
% steps an individual might take towards solving it. Indeed, this is the primary
% step in both popular behavior change programs \cite{twelve-step}, as well as
% more academic considerations \cite{ELM}.

% Various human efforts, over the course of history, have drastically improved the
% comprehensibility of our world. Along with this, the scope of our power to alter
% our world has increased dramatically. Unfortunately, these alterations are not
% always for the better, as is the case with global climate change. There is broad
% agreement that anthropogenic (human caused) climate change is currently and will
% continue to have negative consequences on both human and other forms of life on
% our planet (for example, about 97\% of publishing climate scientists hold this
% view). Certainly, some may say, the planet will endure. But, it seems wise
% to proceed with some concern towards ensuring the survival of those organisms
% and species we hold most dear.

% Certainly, the scope of climate change cognition is far too broad for a research
% project of only a few semesters. As such, I will focus on a handful of issues
% that are of interest from the point of view of a cognitive theory of learning.
% Simultaneously, we maintain an educational point of view that entails a focus on
% variables that we might control. I assume a pragmatic sense of ``poor'' and
% ``good'' cognition regarding climate change and related conceptual domains. The
% goal will be to obtain an understanding that allows us to shift individuals from
% the former to the latter. Roughly speaking, ``good'' cognition would allow
% people to reason more accurately and be more robust to the problematic
% arguments they are likely to encounter in our current political landscape.
% Specific features of such cognition might include:
% 
% \begin{enumerate}
% \item Reasoning with \emph{evidence}. In particular, the use of specific,
% quantitative information (as discussed in section~\ref{sec:ndi} below).
% \item Fluency with models used to explain and predict climate change.
% \item Skeptical evaluation of evidence offered by others.
% \item Ability to connect one's personal values and beliefs to policy
% preferences.
% \end{enumerate}

\section{Science and Numeracy Education for Climate Change}
\label{sec:science-ed}

%%% This should be the end of the chapter
Taken more critically, the above-cited work supports the claim that \emph{absent
proper guidance}, individuals will not arrive at a complete understanding or
acceptance of the current scientific consensus.  Indeed,
\textcite{lewandowsky_pivotal_2013} show that offering climate scientists'
consensus boosts anthropogenic climate change acceptance.  Thus, as we'll argue
below, a more sensible position is likely that science education is hard, but if
done correctly, it can have useful effects.  It's not hard to \emph{want} to
take an educational approach to the climate problem---it's likely the most
democratic approach we can devise! In the face of the rhetoric described above,
educators may already be witholding scientific materials for fear that they will
be polarizing. In the studies presented in the following chapters, we'll see
that even a small amount of true information can quickly act as a cognitive
``lever'' to enhance one's understanding and perspective on climate change.

Note both that new knowledge often facilitates societal shifts and that science
``education'' has historically driven major social changes—from heliocentrism
replacing church doctrine to the acceptance of a tobacco-cancer link in spite of
industry obfuscations. (We offer more such germane evidence below.) These
data-driven shifts demonstrate how sociologists and social psychologists who
hold the stasis view must be incorrect or overly pessimistic. Whether or not
they realize it, theorists are haggling over speed, and some nations learn
(e.g., to accept evolution or climate change; \cite{ranney_why_2012}) faster than others.
Of course, learning or acting too slowly can exacerbate existing problems.

We partially agree, though, with those who critique a ``knowledge deficit'' view
of public attitudes \parencite[cf.][]{dickson_case_2005}. Arbitrary or
propaganda-like information need not drive one toward a more empirically
supported view. Similarly, as argued by both \textcite{kahan_polarizing_2012}
and \textcite{ranney_why_2012}, individuals assessment of the risks of GW will
be based on an interaction between individuals' knowledge and values. We see the
problem as a wisdom deficit, for which cognitively sophisticated educators can
provide the tools that help the public better evaluate the evidence and make
choices that match their values. (See \cite{lewandowsky_misinformation_2012} for
a fine discussion of such tools, particularly the correction of misinformation.)
We believe that the findings described here will demonstrate that a well
considered educational approach can avoid triggering a “conceptual immune
response” by avoiding a clash with individuals' values---focusing on surprising
but non-controversial information. Such interventions may prove critical for
public engagement.

% In an ideal world, the structure of this thesis would harken back to early
% psychophysical research. There are a number of factors that we are certain will
% induce surprise, acceptance of novel ideas and other forms of learning. A
% lovely question could take the form of ``How many units of surprise yield so and
% so units of attitudinal shift?'' or ``When matched for identical amounts of
% cognition, what is the relative effect of numerically-grounded evidence vs.\
% emotionally charged evidence?'' As is plain to see, such precision is well
% beyond the current state of the art. Thus, I will focus primarily on categorical
% differences between educational interventions and participants memories and
% explanations. Below, I lay out a number of issues that figure heavily into the
% selection of these categories.

Clear successes have been observed with education for controversial science.
Evolution is one oft-cited example of a polarizing science topic, but
\textcite{shtulman_learning_2008} report that increases in evolution knowledge
do indeed boost acceptance. It may be that education must specifically
cover critical gaps in students' knowledge. While we should act with caution in
comparing GW and evolution education, it is relevant to notice here that both
topics require
vastly larger views of time as compared to many other sciences. The longest
sample of atmospheric greenhouse gasses and global temperature goes back an
impressive 800,000 years.  Impressive, that is, until you consider that most
major phyla emerged around 530 million years ago. Thus, arguments regarding time
scale or complexity of systems can be readily addressed with successful
evolution education in addition to the kinds of experiments we'll see below.
% Another interesting point of comparison would be Plate Tectonics (which
% apparently is also quite controversial in the south)
\textcite{sinatra_promoting_2012} also demonstrate that GW attitudes can be shifted
with textual materials, though in this case with a more “persuasive” essay. As
we'll see in Chapter~\ref{chap:evilndi}, however, even individual's with high
acceptance levels may be dramatically swayed by only a dash of misleading
factual information. 

\subsection{The Numerically Driven Inferencing (NDI) Paradigm}
\label{sec:ndi}

% It would probably make sense to highlight a more social psych citation here,
% as it would be more germane to climate change. Perhaps even one of the more
% recent climate change polarization articles?

In addition to the arguments offered above, our laboratory has provided
arguments and many experimental findings that run counter to ``polarization'' or
stasis in response to science or numerical information.  Our group has observed
policy shifts regarding many other social issues (e.g., abortion and
immigration) with a single number/statistic
\parencite{garcia_de_osuna_qualitative_2004,munnich_policy_2003,ranney_designing_2008}.
Below, we offer more experimental results that counter the stasis view, and we
explain the different results, in part, by noting that we include a full
spectrum of participants, rather than filtering for those who are already
relatively extreme.


NDI procedures \parencite[introduced by][]{ranney_numerically_2001_fixed} provide an
approach to changing conceptions, attitudes and even behaviors with quite
minimalist interventions (e.g., providing estimators with a single, critical,
highly germane, feedback statistic, cf. \cite{rinne_estimation_2006}). As
with the climate change literature reviewed above, the
education and social psychology literature provide multiple examples of failures
to elicit conceptual change. For example, \textcite{chi_commonsense_2005} describes
an intervention in which only 1 in 100 eighth-graders were able to shift to a
correct conceptual model of diffusion. Similar examples are available in a
variety of literatures \parencite[cf.][]{disessa_what_1998, lord_biased_1979}.
Certainly, there are marked differences between the above mentioned approaches to
conceptual change. For the purposes of the current effort, we will focus our
attention on those approaches that have been successful (namely, NDI and
targeted science education approaches).

One of the elements of the NDI program, The EPIC procedure, represents an
intervention that is relatively compact and well specified. More importantly,
EPIC has been shown to induce long-lasting conceptual change
\parencite[e.g.,][]{ranney_designing_2008}, as evidenced by increased accuracy on estimations
up to 12 weeks later \parencite{munnich_longevities_2005}.  In the EPIC procedure,
participants engage with real-world numerical facts that bear on a societal
issue, such as abortion, criminal justice, the environment, etc.
\parencite[e.g.,][]{garcia_de_osuna_qualitative_2004,munnich_policy_2003}.  
People often poorly
estimate these quantities, such that the true values are surprising to many
individuals, and experimental research on NDI has provided the basis for
successful classroom curricula for both high school students and graduate
students in journalism
\parencite{munnich_numerically-driven_2004,ranney_designing_2008}.  During the EPIC
procedure, participants:

\begin{enumerate}
\item Provide an \textbf{Estimate} for each policy-relevant item,
\item State what they would \textbf{Prefer} each quantity to be, 
\item Receive actual quantities as feedback to \textbf{Incorporate} (as new
``Information''), and 
\item Indicate whether their preferences have \textbf{Changed} upon receiving feedback.
\end{enumerate}

Work that we'll see in Chapter~\ref{chap:two} sheds light on the cognitive
components of a simpler Estimate-Inform procedure. Moving forwards, expanding
into an exploration of Preference (in the form of climate-relevant attitudes)
allows for a more complete exploration of the effects of such interventions.

\section{Conceptual and \texorpdfstring{``Less Conceptual''}{``Less Conceptual''} Cognition}
\label{sec:two}

% I am egregiously missing a reference to Carey below

Above, we have seen that a status quo seems to be arising in climate change
communication. Specifically, it is claimed that one should focus on ``less
conceptual'' processing in devising interventions for the public. While there is
little data available on the cognition of these scientists, I would hazard to
guess that part of the appeal of ``less conceptual'' cognition (as with the
presentation of GW relevant risks) is that individuals can interface with such
materials much more readily than with more complex materials such as
with science education. However, while less complex or conceptual materials
may be faster to comprehend than science-related materials, the total amount of 
\emph{learning} or \emph{conceptual change} can be far greater when that
individual engages with concepts with rich connections to their understandings.

In its limit, the conceptual domain is the space of cognitive processes where
everything is connected to everything. Strong examples would include Whorfian
(or neo-Whorfian) theories in which language constrains visual perception
\parencite{boroditsky_does_2001}, or the notion of embodied cognition claims in
which our emotional preferences for spatially arranged items may be guided by
our fluency with our own right or left sides
\parencite{casasanto_embodiment_2009}. A more prosaic example illustrating the
difference between more and less conceptual processing is provided in
\textcite{clark_assembling_2003}, in which learning with pre-existing knowledge
(specifically, encoding known words vs. plausible pseudo-words) lowered demands
on prefrontal and parietal working memory structures.

Our mind is also endowed with a number of special-purpose, relatively stable,
fast, local (encapsulated) or ``hard-wired'' capacities. The ``motor
system''\footnote{There may be more than one motor system, but at least one of
them should serve to illustrate this point.} is an excellent example of this.
Conceptually, our motor experience is simple---we desire an object and simply
reach for it. Under the hood, an enormous number of degrees of freedom are
resolved, satisfying multiple complex constraints all without our awareness.
\textcite{clark_multiple_2010} construct a set of features that roughly describe
the nature of cognitive processing in more or less conceptual modes. I adapt the
table given there for Table~\ref{table:multiple}.  Depending on the needs of a
given behavior, learning (or performance) might be better handled by cognition
of one sort or the other. These criteria echo what is discussed in the decision
making literature \parencite{kahneman_perspective_2003} and cognitive
development literature \parencite{carey_origin_2011}.

\begin{table}
\caption{Features of more or less conceptual processing. Adapted (liberally) from
\protect \textcite{clark_multiple_2010}} 
% Consider referencing Sloman's 2-systems here?
\label{table:multiple}
\centering
\begin{tabu}{XX}
\toprule
\textbf{More conceptual} & \textbf{Less conceptual} \\ \midrule
Large amount of learning per trial that saturates quickly (high gain) &
Small, incremental amount of learning per trial (low gain) \\

Requires extra time, cognitive resources for processing &
Learns automatically without effort \\

Required for contextual learning &
Unimodal or modular learning \\

Accessible to awareness and conscious intention &
Impenetrable to awareness, operates independent of conscious strategies \\

Consolidation processes are enhanced during sleep &
Consolidates off-line with the simple passage of time \\

Ready transfer to related tasks &
Task-specific and inflexible \\

Rational and recollective &
Emotional and intuitive \\
\bottomrule
\end{tabu}
\end{table}

% This was about RTMD, I think.
% Maybe read / cite Hoadley, et al. (1994)?

\subsection{NDI: A Successful Model for Conceptual Change}

% Basic point - people may confuse conceptual change with deliberate conscious
% acceptance (this is an extension of classic models for Semantic memory).
% However, we can have the goal of shifting people's scientific or numeric
% concepts closer to reality without taking the standard approach via episodic
% memory

A fundamental question in cognition concerns the nature of what is learned. Some
well-established psychological learning and memory models
\parencite[e.g.,][]{nadel_memory_1997} might predict that changes in estimation accuracy must
ultimately be mediated by the consolidation of episodic memory. In this case, we
would expect participants' reports of explicit memory for feedback (the ``I'' in
EPIC) to correlate well with improvements in estimation accuracy at subsequent
testing.  This would clearly be learning of a conceptual type.

% At one point I thought this should be moved to the next chapter + final
% concluding chapter. Now?
Recent evidence suggests, however, that pre-existing conceptual structures can
be re-modeled in a highly efficient manner that may not rely as heavily on the
brain structures implicated in episodic memory formation
\parencite{tse_schemas_2007,clark_assembling_2003}. In this case, we might
expect increases in estimation accuracy even when participants report no
(episodic) memory whatsoever for the quantity provided as
feedback---particularly if participants had pre-existing knowledge to support
such learning. There may be multiple routes to re-modeling our conceptual
stores, even when our learning experience is of a somewhat less conceptual
flavor.

Evidence of pre-existing knowledge is indicated by surprise upon receiving
feedback, which implies an incorrect prior expectation regarding the true value.
However, subsequent learning that correlates with surprise might also be
explained by an account involving the emotional impact of the information
\parencite{munnich_surprise_2007,thagard_hot_2006}.  Therefore, it is important to
assess not only surprise, but also whether the surprise had an emotional (i.e.,
less conceptual) character. It may be the case that surprise mediates improved
episodic memory. Alternatively, surprise and the existence of prior knowledge
may operate partly or wholly in parallel---mediating direct changes in semantic
memory.

Most generally, learning may be driven by the actual experience of surprise
\parencite[e.g.,][]{munnich_longevities_2005,kang_wick_2009}.  In addition,
improvements in estimation could be driven by a direct (potentially approximate)
episodic memory of feedback. Thus, it seems useful to query participants'
surprise, and whether it is of a more emotional or conceptual sort (we'll see
examples of this below). In addition, we can probe participants' memory in an
attempt to assess conscious recollective ability. In the end, these processes
are likely overlapping, but it may be possible to differentially drive some
aspects of learning and not others, etc.

\section{Summary}

There is a clear need for the development of educational interventions targeting
climate change acceptance and attitudes. Above, we have seen that there is some
indication that compact, evidence-based interventions may provide notable,
durable shifts in policy-relevant attitudes. In the chapters that follow, we
will more closely examine a set of experiments regarding these sorts of
approaches to climate change cognition. Chapter~\ref{chap:two} reports on an
experiment that illuminates some aspects of the psychological processing of such
information, particularly with respect to the role of surprise and conscious
subsequent memory. Such notions may be central to understanding how a successful
intervention would work. Subsequent chapters will discuss a variety of
interventions specifically applied to GW understanding, and assess impacts on
beliefs and attitudes.


%% Put a summary here?

% \section{Questions}
% 
% To recapitulate, I have laid out a number of categorical distinctions above.
% 
% \begin{enumerate}
% \item There is clear evidence that attitudes and beliefs treated by the RTMD theory
% are predictive of one another. This may be largely a cultural or societal
% artefact, in which case learning would be relatively confined within a
% construct. Or, these relationships may reflect the representational structure of
% these ideas in our minds, in which case we might expect a change in one part of
% that network to have effects elsewhere.
% \item Evidence may be objective and concrete (i.e., it may seem very
% \emph{factual}), or it may seem partisan and/or poorly defined. This feature of
% an argument may have differential effects on how much people are moved upon
% hearing the argument, and the retention of any such changes (or even gradual
% increase, as in the classic ``hypermnesia'' paradigm).
% \item There is strong evidence that cognition may occur in relatively isolated,
% automatic systems, or alternatively in a more integrated or conceptual fashion.
% We can seek to characterize what kind of process occurs during a given learning
% or production (memory or explanation) episode. Given this characterization, we
% can again observe the magnitude and timecourse of any changes that are elicited.
% \end{enumerate}
% 
% I will evaluate the following hypotheses.
% 
% \begin{enumerate}
% \item Increased knowledge and understanding will yield greater acceptance of
% climate change (and similarly with evolution).
% \item Emotional engagement will play a role in climate change acceptance or
% rejection, as well as enhancing learning.
% \item Based on the relative success of NDI interventions, I expect that
% numerically-grounded or mechanistic arguments will result
% in more durable shifts, both against the passage of time, and against
% interference or agnotology.
% \item Alternatively, methods of persuasion that appeal to emotion, non-quantitative
% depictions of negative consequences and ethical arguments may have larger
% immediate effects.
% \item Emotional responses (like emotional surprise) will trigger larger shifts
% in attitudes, or increase the likelihood of change in attitudes and learning in
% general.
% \item No single point of entry will be necessary - changing behavior, appeal to
% emotions or provision of rational argument should all be sufficient to have some
% effect on their own.
% \item Multiple methods of engagement in parallel should interact to yield
% greater shifts / learning than the sum of those methods individual effects.
% \item Other attitudes (as in RTMD) will differentially enhance or dampen changes in
% climate change cognition.
% \end{enumerate}
% 
% \section{Some notes on graphical models}
% 
% Throughout this document, I'll use graphical models to supplement tables and
% textual descriptions.
% 
% TODO: Write more about this here!
