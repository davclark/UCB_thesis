\chapter{Introduction} \label{c.intro}

% \epigraph{Let us, then, omit the conjectures of men who know not what they
% say, when they speak of the nature and origin of the human race... They are
% deceived, too, by those highly mendacious documents which profess to give the
% history of many thousand years, though, reckoning by the sacred writings, we
% find that not 6000 years have yet passed.}{St. Augustine, c.  397 CE}

Abstract goes here?

Here's my current take on general issues:

Look at prior to interventions: we have evidence from e.g., Sarah's survey that
mechanism knowledge does indeed correlate with acceptance and willingness to
sacrifice. We look to see if this holds in our mechanism and pro NDI surveys.
Perhaps also in anti-NDI.

We will primarily seek to understand approaches to supporting---and, in one case,
undermining---acceptance and concern regarding global climate change.

\section{We're All Going to Die}
% \section{The problem}
% TODO: This section should be a focused discussion motivating that climate
% change is a REAL and BEHAVIORAL problem.

Apart from those lucky few deathless microbes who might catch a ride out of this
solar system before the sun goes extinct, all organisms on earth today will
ultimately find themselves dead. Even us. This is, in my opinion, no great cause
for alarm, though arguments to this effect fall outside of the scope of this
dissertation. (So, on this point, I must refer you to other sources of your own
choosing.) But I hope I needn't argue the basic fact---a fact that I believe is
necessary for clarity in what we might (or should) possibly hope to accomplish
in mitigating climate change.

% Here describe extinctions, food shortages, military concerns about world
% security, etc.
% Maybe also some John D'Agata? Or perhaps in concluding chapter?

Include an overview of the opposition - ranging from the general anti-science
(Oreskies and Conway) to the economic (Lomberg), Muller, etc.

% This is from CogSci 2013

\subsection{Climate Change as a Behavioral Problem}

Our atmosphere's carbon dioxide (CO2) concentration is higher now than in any of
the past 15 million years (World Bank, 2012). Global warming (``GW'') akin to
recent trends last occurred over 17 million years ago, when a 3-4°C gain over
1,500,000 years occurred. Standard models show that continuing our current
behavior will yield similar warming in just 100 years. In such previous warming
periods, widespread extinctions occurred. With warming being on a 10,000+ times
faster-than-otherwise timescale,
% Insert ref to Harte & Harte, 2008 here
the biological systems we depend upon (e.g., for food) will clearly be severely
impacted (Barnosky, 2009). Nearly all climate researchers have concluded that
the problem is urgent and anthropogenic (i.e., essentially 100\%
human-caused)---and as it is thus behavioral, it will be ``solved'' only by
changes in human behavior. The IPCC (Intergovernmental Panel on Climate Change)
and Skeptical Science have assembled and disseminated the scientific consensus
on GW, but, sadly, the U.S. public remains divided on both the existence and the
cause of climate change (cf. Hoffman, 2011).

\section{Americans are weird}

What's essential about Ranneys theory - the way science ends up in opposition to
other things

Also note story of stuff re: american exceptionalism

\textcite{shtulman-evo-2010-ish}


\section{Cognitive Considerations of Climate Communication Strategies}

% Insert here a more thorough description of the knowledge deficit camp, find
% stuff I sent to Michael as part of the CogSci, not particularly Hansen and
% others who introduce the "knowledge deficit" terminology
A group of climate communication researchers, oddly, suggests that educational
ventures would be of little or no help.  Kahan et al. (2012) found (through
correlational means) that, for the U.S. (a high per-capita carbon user), direct
cognitive approaches (including numeracy and science education) seem to solidify
biased views---reinforcing a kind of cognitive stasis for GW attitudes.
Similarly, McCright and Dunlap (2011) highlight data indicating that the
``education level'' effect on climate belief is moderated by
conservatism/affiliation (with conservative or ``Republican'' GW denial being
slightly positively related, if at all, with education). This (also
correlational) evidence, they claim, disproves a naïve ``knowledge deficit''
view---that is, the view that more education can shift the public's beliefs
toward the scientific consensus about climate change. However, their own work
shows a bifurcation in the kinds of information that liberals and conservatives
tend to receive.
This split leaves open the possibility that well-constructed
interventions may indeed induce conservatives to accept the scientific consensus
(with little challenge to their core values).  

% This is new from me
Taken more critically, the above-cited work supports the claim that \emph{absent
proper guidance}, individuals will not arrive at a complete understanding or
acceptance of the current scientific consensus.
Indeed, Lewandowsky, Gignac, and
Vaughan (2013) show that offering climate scientists' consensus boosts
anthropogenic climate change acceptance.

%TODO: regarding the central argument:
One’s capacity as a philosopher need not be
terribly advanced to infer that disbelief in a problem will likely inhibit any
steps an individual might take towards solving it. Indeed, this is the primary
step in both popular behavior change programs \cite{twelve-step}, as well as
more academic considerations \cite{ELM}.

A more extreme stasis position might be inferred from some social psychology
work: Lord, Ross, and Lepper (1979) found that people with a strong position
tended to polarize further after receiving (albeit not particularly factual)
information that was contrary to their views. Our laboratory has provided
arguments and many experimental findings that run counter to these
``polarization'' and cognitive stasis views: For instance, even a small amount of
true information can quickly act as a cognitive ``lever'' to enhance one's
understanding and perspective on climate change (Ranney et al., 2012a)---and many
other social issues (e.g., abortion and immigration)---and even with a single
number/statistic (Garcia de Osuna, Ranney, \& Nelson, 2004; Munnich et al., 2003;
Ranney et al., 2008). Below, we offer more experimental results that counter the
stasis view, and we explain the different results, in part, by noting that we
include a full spectrum of participants, rather than filtering for those who are
already relatively extreme.

Note both that new knowledge often facilitates societal shifts and that science
``education'' has historically driven major social changes—from heliocentrism
replacing church doctrine to the acceptance of a tobacco-cancer link in spite of
industry obfuscations. (We offer more such germane evidence below.) These
data-driven shifts demonstrate how sociologists and social psychologists who
hold the stasis view must be incorrect or overly pessimistic. Whether or not
they realize it, theorists are haggling over speed, and some nations learn
(e.g., to accept evolution or climate change; Ranney, 2012) faster than others.
Of course, learning or acting too slowly can exacerbate existing problems.

We partially agree, though, with those who critique a ``knowledge deficit'' view
of public attitudes (cf. Dickson, 2005). Arbitrary or propaganda-like
information need not drive one toward a more empirically supported view. We see
the problem as a wisdom deficit, for which cognitively sophisticated educators
can provide the tools that help the public better evaluate the evidence and make
choices that match their values. (See Lewandowsky et al., 2012, for a fine
discussion of such tools, particularly the correction of misinformation.) We
believe that the findings described here will demonstrate that a well considered
educational approach is critical for public engagement.


% Used to think:
% Make this more of a lead-in to the rtmd, NDI and dual-system learning
% Now I think cut RTMD more or less

Various human efforts, over the course of history, have drastically improved the
comprehensibility of our world. Along with this, the scope of our power to alter
our world has increased dramatically. Unfortunately, these alterations are not
always for the better, as is the case with global climate change. There is broad
agreement that anthropogenic (human caused) climate change is currently and will
continue to have negative consequences on both human and other forms of life on
our planet (for example, about 97\% of publishing climate scientists hold this
view). Certainly, some may say, the planet will endure. But, it seems wise
to proceed with some concern towards ensuring the survival of those organisms
and species we hold most dear.

While humans continue to increase our understanding of the world, issues like
climate change appear to be a bit beyond what is readily comprehended by
non-specialists. In particular, there seems to be a trouble with the
comprehension and acceptance of climate change in America. While much of the
developed world accepts anthropogenic climate change as a reality, as of January
2010 only 57\% of individuals surveyed in the United States think global warming
is happening at all. When asked to assume that global warming \emph{is}
happening, only 47\% of the same group of respondants indicated that they
thought it was ``caused mostly by human activities'' \cite[Q47 and Q50
in][]{leiserowitz_climate_2010}.  Presumably, the number of Americans accepting
anthropogenic climate change is somewhat less than this figure. Thus, if we want
to do something about this issue in the context of a democratic society, the
first step is getting a reasonable majority of people to accept that there at
least \emph{may} be a problem \cite{prochaska_toward_1986}.
% Note - Canny thinks the TTM cited here is not terribly relevant
Thus, the overarching question, ``How can ideas from cognitive science help us
improve climate change education?''

Certainly, the scope of climate change cognition is far too broad for a research
project of only a few semesters. As such, I will focus on a handful of issues
that are of interest from the point of view of a cognitive theory of learning.
Simultaneously, we maintain an educational point of view that entails a focus on
variables that we might control. I assume a pragmatic sense of ``poor'' and
``good'' cognition regarding climate change and related conceptual domains. The
goal will be to obtain an understanding that allows us to shift individuals from
the former to the latter. Roughly speaking, ``good'' cognition would allow
people to reason more accurately and be more robust to the problematic
arguments they are likely to encounter in our current political landscape.
Specific features of such cognition might include:

\begin{enumerate}
\item Reasoning with \emph{evidence}. In particular, the use of specific,
quantitative information (as discussed in section~\ref{sec:ndi} below).
\item Fluency with models used to explain and predict climate change.
\item Skeptical evaluation of evidence offered by others.
\item Ability to connect one's personal values and beliefs to policy
preferences.
\end{enumerate}

In an ideal world, the structure of this thesis would harken back to early
psychophysical research. There are a number of factors that we are certain will
induce surprise, acceptance of novel ideas and other forms of learning. A
lovely question could take the form of ``How many units of surprise yield so and
so units of attitudinal shift?'' or ``When matched for identical amounts of
cognition, what is the relative effect of numerically-grounded evidence vs.\
emotionally charged evidence?'' As is plain to see, such precision is well
beyond the current state of the art. Thus, I will focus primarily on categorical
differences between educational interventions and participants memories and
explanations. Below, I lay out a number of issues that figure heavily into the
selection of these categories.

\section{Science and Numeracy education for climate change}

This should be much clearer and more focused! 

Central points: RTMD explores general weirdness of Americans, note relationship
between evolution and global warming acceptance (which will be reported on more
fully below). Also set up the basic “knowledge deficit” view, and explain why we
think we'll do OK: science education can change beliefs, as in Shtulman and
Calabi, and other arguments as exemplified in the 2013 Cog Sci paper.

\section{Reinforced Theistic Manifest Destiny theory}

%% Problematically, Al Gore also talks a lot about melting ice - arguing against
%% the bulk of our pro-GW items!

\citeauthor{ranney_why_inpress} observe that in addition to America's disparity
with peer nations in accepting global warming, Americans are also appreciably
lower in their acceptance of Darwinian evolution. Indeed, the United States
ranks 33rd of 34 peer nations in acceptance of evolution, putting us
squarely between Cypress and Turkey. The ``received view'' is that Americans are
particularly fundamentalist in their acceptance of biblical creation\footnote{I
will focus primarily on the Christian faith, as it is the dominant religion in
the U.S.\ and peer nations of interest. 84\% of Americans practice some form of
Christianity, with Judaism and Islam as the second and third most common at
1.9\% and 1.6\% respectively \cite{wolfram_alpha_faith}.}, 
necessitating the rejection of evolutionary ideas. But, this notion fails to
address the similar pattern observed with acceptance of climate change. In
addition, many of the afore-mentioned peer nations also exhibit high adherence
to Christian faiths.  Likewise, it is not clear that America is sufficiently
more fundamentalist than peer nations to explain all the important variance in
our acceptance of these ideas. 

\begin{figure}[h]
\centering
\includegraphics[width=\textwidth]{rtmd.pdf}
\caption{Ranney's RTMD model. The ``received view'' is emphasized, with
a negative relationship between belief in creation and acceptance of evolution
represented by the dashed line. Solid lines express positive relationships
between these and a network of related attitudes. A causal point of entry is
expressed for ``being a human,'' fearing death and thus, desiring an afterlife.}
\label{fig:rtmd} 
\end{figure}

The central insight of \citeauthor{ranney_accepting_2011} is that what sets the
United States apart from the rest of the world is the relative success it has
experienced over the previous century, called by some ``The American Century.''
Our war efforts have been (or at least seem) enormously successful, our economy
is far larger than any other economy in the world, we are successful in the
Olympics and so on. This leads to a
feeling that ``God is on our side,'' or in other words, it reinforces notions of
manifest destiny via quasi-religious sentiment.


Another general problem identified within RTMD is the fact that individuals
often reject scientific ideas when they are in conflict with their other
attitudes and beliefs. It's as if we are endowed with something of a conceptual
immune system, % Maybe elaborate more in thesis?
comprising religious and nationalistic beliefs in some, and more
scientifically grounded beliefs in others.  But note, there is a sharp distinction between
the kinds of ``emotional responses'' that might be experienced between climate
change and evolution. Specifically, climate change is something that
involves the ethical status of actions that we do every day, both individually
and as a society. Evolution, on the other hand, tends to incohere with
personally held religious beliefs, most directly divine creation, but then by
extension, deity and afterlife.

As an interesting side-note, both evolution and climate science require vastly
larger views of time as compared to many other sciences. The longest sample of
atmospheric greenhouse gasses and global temperature goes back an impressive
800,000 years.  Impressive, that is, until you consider that most major phyla
emerged around 530 million years ago (and have thus been influencing the climate
since that time).

% An interesting point of comparison would be Plate Tectonics (which apparently
% is also quite controversial in the south

RTMD doesn't disagree with other explanations, such as an analysis of
individuals into democrats and republicans. Rather, it provides a more specific
set of proximal relationships between concepts. According to the theory, for
example, acceptance of evolution should have \emph{more} predictive value for
one's global warming acceptance as compared to, say, acceptance of a deity.
Additionally, these relationships may predict related changes we might see as we
modulate individual's attitudes.


% Perhaps mention the inclusion of notions of consumerism?

\section{Reasoning with Numbers \label{sec:ndi}}

% It would probably make sense to highlight a more social psych citation here,
% as it would be more germane to climate change. Perhaps even one of the more
% recent climate change polarization articles?

NDI procedures \cite[introduced by][]{ranney_numerically_2001_fixed} provide an
approach to changing conceptions, attitudes and even behaviors with quite
minimalist interventions (e.g., providing estimators with a single, critical,
highly germane, feedback statistic, cf. \nptextcite{rinne_estimation_2006}).  The
education and social psychology literature provide multiple examples of failures
to elicit conceptual change. For example, \citeauthor{chi_commonsense_2005} describes
an intervention in which only 1 in 100 eighth-graders were able to shift to a
correct conceptual model of diffusion. Similar examples are available in a
variety of literatures \cite[cf.][]{disessa_what_1998, lord_biased_1979}.
Certainly, the are marked differences between the above mentioned approaches to
conceptual change. For the purposes of the current effort, we will focus our
attention on those approaches that have been successful (namely, NDI and related
approaches).

One of the elements of the NDI program, The EPIC procedure, represents an
intervention that is relatively compact and well specified. More importantly,
EPIC has been shown to induce long-lasting conceptual change
\cite[e.g.,][]{ranney_designing_2008}, as evidenced by increased accuracy on estimations
up to 12 weeks later \cite{munnich_longevities_2005}.  In the EPIC procedure,
participants engage with real-world numerical facts that bear on a societal
issue, such as abortion, criminal justice, the environment, etc.
\cite[e.g.,][]{garcia_de_osuna_qualitative_2004_fixed,munnich_policy_2003_fixed}.  
People often poorly
estimate these quantities, such that the true values are surprising to many
individuals, and experimental research on NDI has provided the basis for
successful classroom curricula for both high school students and graduate
students in journalism
\cite{munnich_numerically-driven_2004,ranney_designing_2008}.  During the EPIC
procedure, participants

\begin{enumerate}
\item Provide an \textbf{Estimate} for each policy-relevant item,
\item State what they would \textbf{Prefer} each quantity to be, 
\item Receive actual quantities as feedback to \textbf{Incorporate} (as new
``Information''), and 
\item Indicate whether their preferences have \textbf{Changed} upon receiving feedback.
\end{enumerate}

Work that I'll describe below has examined the cognitive components of a simpler
Estimate-Inform procedure. Moving forwards, expanding into an exploration of
Preference allows for a clear point of connection with the attitudes treated by
the RTMD theory.

\section{Conceptual and \texorpdfstring{``less conceptual''}{``less conceptual''} cognition}
\label{sec:two}

The central point to make here is that the status quo (in climate change
communication) seems to think you should focus on ``less conceptual''
processing. While it's impossible to know given the available data on
scientists, I would hazard to guess that the in-the-moment speed of ``less
conceptual'' processing seduces communicators away from the much faster
\emph{learning} speeds of conceptual processing.

In its limit, the conceptual domain is the space of cognitive processes where
everything is connected to everything. Strong examples would include Whorfian
theories in which language constrains visual perception
\cite{boroditsky_does_2001}, or the notion of embodied cognition claims in which
our emotional preferences for spatially arranged items may be guided by our
fluency with our own right or left sides \cite{casasanto_embodiment_2009}. A
more prosaic example illustrating the difference between more and less conceptual
processing is provided in \citeauthor{clark_assembling_2003}, in which learning with
pre-existing knowledge (specifically, encoding known words vs. plausible
pseudo-words) lowered demands on prefrontal and parietal working memory
structures.

Our mind is also endowed with a number of special-purpose, relatively stable,
fast, local (encapsulated) or ``hard-wired'' capacities. The ``motor
system''\footnote{There may be more than one motor system, but at least one of
them should serve to illustrate this point.} is an excellent example of this.
Conceptually, our motor experience is simple---we desire an object and simply
reach for it. Under the hood, an enormous number of degrees of freedom are
resolved, satisfying multiple complex constraints all without our awareness.
\citeauthor{clark_multiple_2010} construct a set of features that roughly describe
the nature of cognitive processing in more or less conceptual modes. I adapt the
table given there for Table~\ref{table:multiple}.  Depending on the needs of a
given behavior, learning (or performance) might be better handled by cognition
of one sort or the other. These criteria echo what is discussed in the decision
making literature \cite{kahneman_perspective_2003}.

\begin{table}
\centering
\begin{tabular}{p{0.45\textwidth}p{0.45\textwidth}}
\textbf{More conceptual} & \textbf{Less conceptual} \\ \hline \hline

Large amount of learning per trial that saturates quickly (high gain) &
Small, incremental amount of learning per trial (low gain) \\
\hline

Requires extra time, cognitive resources for processing &
Learns automatically without effort \\
\hline

Required for contextual learning &
Unimodal or modular learning \\
\hline

Accessible to awareness and conscious intention &
Impenetrable to awareness, operates independent of conscious strategies \\
\hline

Consolidation processes are enhanced during sleep &
Consolidates off-line with the simple passage of time \\
\hline

Ready transfer to related tasks &
Task-specific and inflexible \\
\hline

Rational and recollective &
Emotional and intuitive \\
\hline
\end{tabular}
\caption{Features of more or less conceptual processing. Adapted (liberally) from
\protect \citeauthor{clark_multiple_2010}} 
% Consider referencing Sloman's 2-systems here?
\label{table:multiple}
\end{table}

\subsection{Relating to RTMD}

Given these notions, we can revisit the RTMD theory and suggest that perhaps the
relationships in that theory have a less fully conceptual flavor.  That is, I
suspect individuals don't explicitly consider, say, their feelings of national
pride when cognizing about evolution.  This may be similar to a subjective
experience we've likely all had. We can learn lots of great well-reasoned things
about someone we don't like or accept, and yet we may come to like or accept
them only slowly (or not at all). It's not a completely rational process!

So, interventions that operate more on non-conceptual aspects of, say,
nationalism (i.e., via emotionally charged items) may enhance the effect of
these relationships (i.e., increase correlations between them, or cause shifts in
related attitudes) as compared to shifts elicited by more objective items (such
as a mechanistic explanation of climate change).  On the other hand, it may be
that ideas don't support or detract from one another unless they are both
consciously held in memory at the same time. If this is the case, bringing these
relationships to mind may strengthen RTMD related effects.
% Maybe read / cite Hoadley, et al. (1994)?

\subsection{Relating to NDI}

A fundamental question in cognition concerns the nature of what is learned. Some
well-established psychological learning and memory models
\cite{nadel_memory_1997} might predict that changes in estimation accuracy must
ultimately be mediated by the consolidation of episodic memory. In this case, we
would expect participants' reports of explicit memory for feedback (the ``I'' in
EPIC) to correlate well with improvements in estimation accuracy at subsequent
testing.  This would clearly be learning of a conceptual type.

Recent evidence suggests, however, that pre-existing conceptual structures can
be re-modeled in a highly efficient manner that may not rely as heavily on the
brain structures implicated in episodic memory formation
\cite{tse_schemas_2007,clark_assembling_2003}. In this case, we might expect
increases in estimation accuracy even when participants report no memory
whatsoever for the quantity provided as feedback---particularly if participants
had pre-existing knowledge to support such learning.

Evidence of pre-existing knowledge is indicated by surprise upon receiving
feedback, which implies an incorrect prior expectation regarding the true value.
However, subsequent learning that correlates with surprise might also be
explained by an account involving the emotional impact of the information
\cite{munnich_surprise_2007,thagard_hot_2006}.  Therefore, it is important to
assess not only surprise, but also whether the surprise had an emotional (i.e.,
less conceptual) character. It may be the case that surprise mediates improved
episodic memory. Alternatively, surprise and the existence of prior knowledge
may operate partly or wholly in parallel---mediating direct changes in semantic
memory.

Most generally, learning may be driven by the actual experience of surprise
\cite[e.g.,][]{munnich_longevities_2005}.  In addition, improvements in estimation could be
driven by a direct (potentially approximate) episodic memory of feedback. Thus,
it seems useful to query participants' surprise, and whether it is of a more
emotional or conceptual sort. In addition, we can probe participants memory in
an attempt to assess conscious recollective ability. In the end, these processes
are certainly overlapping, but it may be possible to differentially drive some
aspects of learning and not others, etc.

\section{Summary}

There is a clear need for the development of educational interventions targeting
climate change acceptance and attitudes. Above, we have seen that there is some
indication that compact, evidence-based interventions may provide notable,
durable shifts in policy-relevant attitudes. In the chapters that follow, we
will more closely examine a set of experiments regarding these sorts of
approaches to climate change cognition, some completed and some proposed. These
experiments will illuminate some aspects of the psychological processing of such
interventions that appear central to understanding how a successful intervention
would work.

\section*{Acknowledgments}

Probably won't do acknowledgements in the intro, but this is a good placeholder.

%% Put a summary here?

% \section{Questions}
% 
% To recapitulate, I have laid out a number of categorical distinctions above.
% 
% \begin{enumerate}
% \item There is clear evidence that attitudes and beliefs treated by the RTMD theory
% are predictive of one another. This may be largely a cultural or societal
% artefact, in which case learning would be relatively confined within a
% construct. Or, these relationships may reflect the representational structure of
% these ideas in our minds, in which case we might expect a change in one part of
% that network to have effects elsewhere.
% \item Evidence may be objective and concrete (i.e., it may seem very
% \emph{factual}), or it may seem partisan and/or poorly defined. This feature of
% an argument may have differential effects on how much people are moved upon
% hearing the argument, and the retention of any such changes (or even gradual
% increase, as in the classic ``hypermnesia'' paradigm).
% \item There is strong evidence that cognition may occur in relatively isolated,
% automatic systems, or alternatively in a more integrated or conceptual fashion.
% We can seek to characterize what kind of process occurs during a given learning
% or production (memory or explanation) episode. Given this characterization, we
% can again observe the magnitude and timecourse of any changes that are elicited.
% \end{enumerate}
% 
% I will evaluate the following hypotheses.
% 
% \begin{enumerate}
% \item Increased knowledge and understanding will yield greater acceptance of
% climate change (and similarly with evolution).
% \item Emotional engagement will play a role in climate change acceptance or
% rejection, as well as enhancing learning.
% \item Based on the relative success of NDI interventions, I expect that
% numerically-grounded or mechanistic arguments will result
% in more durable shifts, both against the passage of time, and against
% interference or agnotology.
% \item Alternatively, methods of persuasion that appeal to emotion, non-quantitative
% depictions of negative consequences and ethical arguments may have larger
% immediate effects.
% \item Emotional responses (like emotional surprise) will trigger larger shifts
% in attitudes, or increase the likelihood of change in attitudes and learning in
% general.
% \item No single point of entry will be necessary - changing behavior, appeal to
% emotions or provision of rational argument should all be sufficient to have some
% effect on their own.
% \item Multiple methods of engagement in parallel should interact to yield
% greater shifts / learning than the sum of those methods individual effects.
% \item Other attitudes (as in RTMD) will differentially enhance or dampen changes in
% climate change cognition.
% \end{enumerate}
% 
% \section{Some notes on graphical models}
% 
% Throughout this document, I'll use graphical models to supplement tables and
% textual descriptions.
% 
% TODO: Write more about this here!
