\graphicspath{{conclusion/}}

\chapter{Conclusion}
\label{chap:conclusion}

We have provided an evidential medley that effectively disconfirms the notion
that climate change-relevant knowledge and attitudes are locked in cognitive
stasis. Moreover, contrary to those who over-problematize a ``knowledge deficit''
(or ``information deficit'') approach to climate change communication, we see a
``wisdom deficit.'' Here (and in Ranney et al., 2012a) we have considerably
un-problematized it with the ``cognitive levers'' of these interventions. In
contrast, it is unlikely that offering either an ill-structured list of
uncompelling facts to an unprepared mind or thinly veiled rhetoric (cf. Lord,
Ross, \& Lepper, 1979) will notably alter beliefs or behaviors––especially about
the difficult topic of climate change. Rather, one must be sensitive to specific
(mis)understandings that may be relevant to a learner grappling with a domain.
Ultimately, we will likely need to engage virtually all people, assisting them
in connecting their long-term values to the long-term effects of their
behaviors.

In Study 2, we also showed, disturbingly, that one can readily erode climate
change acceptance with misleading, cherry-picked numbers. We can think of no
better protection against such ``evil'' interventions than to provide the context
necessary to recognize them for the clever misinformation that they are. Such
prophylactic interventions may represent promising targets for further research
and educational initiatives (cf. Lewandowsky et al., 2012).

We are currently studying ways of disseminating the information that we have
found to elicit worthwhile cognitive and belief changes.  For instance, we are
producing on-line instructional materials (e.g., videos) that can widely convey
both global warming’s mechanism and the statistics that reflect the scientific
consensus of climate change—so the public can join that consensus.

We have shown above that on-line survey interventions, brief curricula, and
classroom lessons can have a marked and persistent effect on knowledge,
understanding, beliefs, and attitudes about climate change. In spite of
arguments to the contrary, some simple cognitively-informed interventions might
be fundamental in building the resolve to tackle global climate change.

\section{Demographics}

While a random sample of workers on Mechanical Turk yeilded a relatively
Democratic and GW-accepting population, we were still able to observe
conservatives within this sample.

There may have been some self-selection at play here. But, given the applied
nature of this work, self-selection for this experiment likely reflects
self-selection for engagement with this material on-line. This would do little,
then, to erode the claim that on-line science and numeracy interventions can
have a net positive impact on individuals. Note also that the “polarization”
claim seems highly untenable in the face of results from
Chatper~\ref{chap:evilndi}.

One demographic oddity worth noting is our lack of self-identified “protestants”
from our samples recruited from Mechanical Turk. Certainly, one might expect
evangelicals would avoid materials such as those presented in the studies above.
But there remains a large, relatively liberal segment of the protestant
community. As it stands, this point remains a mystery, and might perhaps be the
subject of further study. Again, however, I would emphasize that applied
research such as presented in this dissertation should focus on populations one
is likely to engage with a real-world educational intervention!

\section{Summary}

Going back to foundational thinkers in [American] education (Dewey?). Is there
value in educating individuals about climate change? In part, it depends on who
you are. If you are a science educator, I hardly needed to complete this
research to tell you that the answer is an emphatic YES! If you are seeking to
influence behavior or policy, however, it is a complex task. But in contrast to
the dominant view, it seems that science education might at least push us in the
right direction (even if it, alone, is not the most effective or efficient route
to conservation). But in the end, if it turns out that we are indeed educable in
a meaningful way, there is perhaps more value in conserving the environment that
sustains us than if we are mere automata to be shoved around with propaganda!

At the highest level, the question becomes: what would we have govern us as a
society at the highest level? Superstition, the preservation of national power,
market forces, or our best shot at objective truth? As we've seen in some of the
discussion above, these approaches are intertwined. But I hope I've made it
clear that apart from philosophical arguments in it's favor, science education
seems to be an effective approach to tackling the behavioral problem of climate
change.
