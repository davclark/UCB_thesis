\graphicspath{{conclusion/}}

\chapter{Conclusion}
\label{chap:conclusion}

We have provided an evidential medley that effectively disconfirms the notion
that climate change-relevant knowledge and attitudes are locked in cognitive
stasis. Moreover, contrary to those who over-problematize a ``knowledge deficit''
(or ``information deficit'') approach to climate change communication, we see a
``wisdom deficit.'' Here (and in Ranney et al., 2012a) we have considerably
un-problematized it with the ``cognitive levers'' of these interventions. In
contrast, it is unlikely that offering either an ill-structured list of
uncompelling facts to an unprepared mind or thinly veiled rhetoric (cf. Lord,
Ross, \& Lepper, 1979) will notably alter beliefs or behaviors––especially about
the difficult topic of climate change. Rather, one must be sensitive to specific
(mis)understandings that may be relevant to a learner grappling with a domain.
Ultimately, we will likely need to engage virtually all people, assisting them
in connecting their long-term values to the long-term effects of their
behaviors.

In Study 2, we also showed, disturbingly, that one can readily erode climate
change acceptance with misleading, cherry-picked numbers. We can think of no
better protection against such ``evil'' interventions than to provide the context
necessary to recognize them for the clever misinformation that they are. Such
prophylactic interventions may represent promising targets for further research
and educational initiatives (cf. Lewandowsky et al., 2012).

We are currently studying ways of disseminating the information that we have
found to elicit worthwhile cognitive and belief changes.  For instance, we are
producing on-line instructional materials (e.g., videos) that can widely convey
both global warming’s mechanism and the statistics that reflect the scientific
consensus of climate change—so the public can join that consensus.

We have shown above that on-line survey interventions, brief curricula, and
classroom lessons can have a marked and persistent effect on knowledge,
understanding, beliefs, and attitudes about climate change. In spite of
arguments to the contrary, some simple cognitively-informed interventions might
be fundamental in building the resolve to tackle global climate change.

