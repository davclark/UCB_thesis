\graphicspath{{conclusion/}}

\chapter{Conclusion}
\label{chap:conclusion}

We have provided an evidential medley that effectively disconfirms the notion
that climate change-relevant knowledge and attitudes are locked in cognitive
stasis. Moreover, contrary to those who over-problematize a ``knowledge
deficit'' (or ``information deficit'') approach to climate change communication,
we see a ``wisdom deficit.'' We have demonstrably and considerably
un-problematized it with the ``cognitive levers'' of the interventions described
in the previous chapters. In contrast, it is unlikely that offering either an
ill-structured list of uncompelling facts to an unprepared mind or thinly veiled
rhetoric \parencite[cf.][]{lord_biased_1979} will notably alter beliefs or
behaviors---especially about the difficult topic of climate change. Rather, one
must be sensitive to specific (mis)understandings that may be relevant to a
learner grappling with a domain.  Ultimately, we will likely need to engage
virtually all people, assisting them in connecting their long-term values to the
long-term effects of their behaviors.

In Chapter~\ref{chap:evilndi}, we also showed, disturbingly, that one can
readily erode climate change acceptance with misleading, cherry-picked numbers.
We can think of no better protection against such ``evil'' interventions than to
provide the context necessary to recognize them for the clever misinformation
that they are. Such prophylactic interventions may represent promising targets
for further research and educational initiatives
\parencite[cf.][]{lewandowsky_misinformation_2012}.

Our research group is currently studying ways of disseminating the information
that we have found to elicit worthwhile cognitive and belief changes.  For
instance, we are producing on-line instructional materials (e.g., videos
available on \url{http://HowGlobalWarmingWorks.org}) that
can widely convey both global warming’s mechanism and the statistics that
reflect the scientific consensus of climate change—so the public can join that
consensus.

We have shown above that on-line survey interventions, brief curricula, and
classroom lessons can have marked and persistent effects on knowledge,
understanding, beliefs, and attitudes about climate change. In spite of
arguments to the contrary, some simple cognitively-informed interventions might
be fundamental in building the resolve to tackle global climate change.

\section{On the Structure of Successful Interventions}

% Mention test-enhanced learning, and what I wrote to Rich:

One of the clearer answers in the educational research literature is the
superiority (for retention) of “test-enhanced learning”
\parencite{roediger_test-enhanced_2006}. The basic idea is that one should
always try to answer a question rather than simply studying the answer. It
should come as no surprise then, that in our own work we've found enhancement
in our interventions from eliciting an answer from the participant prior to
revealing the correct answer (for example, the enhanced surprise seen in our
sandwich group in Section~\ref{sec:mech-classroom}). Thus, it should not seem
controversial that in the experiments above, we have often defaulted to a
pretest without a thorough exploration of the effects of leaving one off. If one
is interested in developing effective interventions, including a pretest is
likely a good starting point!

Regarding the timing of a posttest, while some of our studies in
Chapter~\ref{chap:mechanism} included a delayed test, they \emph{all} included
an immediate posttest. This was more for convenience, and a hedge against rapid
forgetting (at least we might observe some immediate effects, even if they were
lost later---though fortunately, they were not!). Looking at the work presented
in Chapter~\ref{chap:two}, as well as other literature on optimal timing of
practice \parencite[e.g.,][]{cepeda_optimizing_2009}, it is likely that an
immediate posttest is probably \emph{less} effective than a delayed posttest for
long-term retention. Any posttest, however, will likely improve retention.
Clearly, determining the optimal timing for a posttest (or several) remains an
open question for further research. In practice, however, one is likely to be
subject to practical constraints. As such, I would recommend an immediate
pretest, as well as a posttest that occurs after as much of a reasonable lag as
one can build into a curriculum. Critically, as one moves to different media
(such as the video-based materials at \url{http://HowGlobalWarmingWorks.org}),
the best timecourse for practice and retention may differ markedly.

\section{A Note on Demographics}

While a random sample of workers on Mechanical Turk yielded a relatively
Democratic and GW-accepting population, we were still able to capture some
conservatives within this sample. As we've noted in the preceding chapters,
liberal skew on Mechanical Turk seems to be the norm
\parencite{richey_how_2012}. Efforts should be made to evaluate similar science
education efforts directed at communities where conservatives are
better-represented. There may also have been some self-selection at
play in our online studies. But, given the applied nature of this work,
self-selection for this experiment likely reflects self-selection for engagement
with this material online. This would do little, then, to erode the claim that
online science and numeracy interventions can have a net positive impact on
individuals. Note also that the “polarization” claim seems highly untenable in
the face of results from Chapter~\ref{chap:evilndi} (given how UC Berkeley
undergraduates are from a liberal pool of Americans).

\section{On Retrospective Reports of Surprise}

It is also interesting to note the different amounts of surprise reported for
two numbers in the interventions in Chapter~\ref{chap:prondi} versus the
surprise reported for the 400 words in Chapter~\ref{chap:mechanism}.
Specifically, while the 400 words \emph{contains} two of those numbers,
individuals report \emph{less} surprise to the textual mechanistic description.
An interesting analogy is provided by the account of retrospection on
colonoscopies in \textcite{kahneman_perspective_2003}. Here, participants report
retrospective pain that is a weighted combination of their peak experience of
pain, and the pain at the end of the procedure. Thus, while participants may
experience a blip of surprise in the 400 words, the fact that it ends with a
summary of the preceding information almost certainly means it ends on an
unsurprising note.  Thus, participants \emph{may} in fact experience comparable surprise
in both styles of intervention, but when surprise is reported only after the
unsurprising conclusion of the 400 words, this may drive retrospective surprise
down.

Less theoretically interesting, perhaps, but still practically useful, is that
we've seen above that different formulations of surprise may be more relevant to
different forms of information. In particular, participants appear more willing
to say that they were “surprised or embarrassed at their own lack of knowledge”
for our mechanistic explanation (as seen in the study in
Section~\ref{sec:mech-mturk}), while they were more willing to report straight
surprise at numerical information (as seen in the study in
Section~\ref{sec:pro-mturk}). Use of the appropriate question may allow for
better assessment of individual variability in a given intervention. Some care
should be exercised, as certain forms of question appear to be rated uniformly
high (or low) for a given participant (as seen in
Figure~\ref{fig:CCO-prondi-surp-corr}). Such differences might be hashed out in
the development of more refined assessments for future studies.


% \section{An Informal Comparison of Effect Sizes}

\section{Summary and Recommendations}

Is there value in educating individuals about climate change? In part, it
depends on who you are. If you are a science educator, I hardly needed to
complete this research to tell you that the answer is an emphatic \emph{yes}.
But importantly, by using surprising information that stays close to verifiable
facts, you may likely avoid problems with polarization. If you are seeking to
influence behavior or policy, however, it is a complex task.  But in contrast to
the view that climate education is dangerously polarizing, it seems that science
education might at least push us in the right direction (even if it, alone, is
not the most effective or efficient route to conservation). But in the end, if
it turns out that we are indeed educable in a meaningful way, there is perhaps
more value in conserving the environment that sustains us than if we were mere
automata to be shoved around with propaganda!

% Maybe put a reference to CRED / Elke Weber on finite pool of worry, and the
% kinds of problems we run into with Lomberg and even the Harte vs. Schrag axis.
% In the end, this stuff _must_ be informed by science. 

At the highest level, the question becomes, what would we have govern us as a
society at the highest level: Superstition, the preservation of national power,
market forces, or our best shot at objective truth? As we've seen in some of the
discussion above, these approaches are intertwined. But I hope I've made it
clear that, apart from philosophical arguments in it's favor, science education
seems to be an effective approach to tackling the behavioral problem of climate
change.
