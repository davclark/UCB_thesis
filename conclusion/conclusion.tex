\graphicspath{{conclusion/}}

\chapter{Conclusion}
\label{chap:conclusion}

We have provided an evidential medley that effectively disconfirms the notion
that climate change-relevant knowledge and attitudes are locked in cognitive
stasis. Moreover, contrary to those who over-problematize a ``knowledge
deficit'' (or ``information deficit'') approach to climate change communication,
we see a ``wisdom deficit.'' We have demonstrably and considerably
un-problematized it with the ``cognitive levers'' of the interventions described
in the previous chapters. In contrast, it is unlikely that offering either an
ill-structured list of uncompelling facts to an unprepared mind or thinly veiled
rhetoric \parencite[cf.][]{lord_biased_1979} will notably alter beliefs or
behaviors---especially about the difficult topic of climate change. Rather, one
must be sensitive to specific (mis)understandings that may be relevant to a
learner grappling with a domain.  Ultimately, we will likely need to engage
virtually all people, assisting them in connecting their long-term values to the
long-term effects of their behaviors.

In Chapter~\ref{chap:evilndi}, we also showed, disturbingly, that one can
readily erode climate change acceptance with misleading, cherry-picked numbers.
We can think of no better protection against such ``evil'' interventions than to
provide the context necessary to recognize them for the clever misinformation
that they are. Such prophylactic interventions may represent promising targets
for further research and educational initiatives
\parencite[cf.][]{lewandowsky_misinformation_2012}.

Our research group is currently studying ways of disseminating the information
that we have found to elicit worthwhile cognitive and belief changes.  For
instance, we are producing on-line instructional materials (e.g., videos
available on \url{http://HowGlobalWarmingWorks.org}) that
can widely convey both global warming’s mechanism and the statistics that
reflect the scientific consensus of climate change—so the public can join that
consensus.

We have shown above that on-line survey interventions, brief curricula, and
classroom lessons can have marked and persistent effects on knowledge,
understanding, beliefs, and attitudes about climate change. In spite of
arguments to the contrary, some simple cognitively-informed interventions might
be fundamental in building the resolve to tackle global climate change.

\section{On the Structure of Successful Interventions}

% Mention test-enhanced learning, and what I wrote to Rich:

% To quickly answer your question / confusion - all particpants in climate-related interventions did an immediate posttest. However, the results I obtained in Chapter 3 speak to this general issue (which you might recall as exploring the effect of immediate or delayed feedback on the impact of a broad range of policy-relevant numbers). In particular, those results indicated no effect of lag (or if anything, a modest gain when participants have more delayed followup testing).
% 
% So, I'll fold that in to the discussion. Makes the whole dissertation sound a bit more coherent (though if I tried to publish this, I might ditch the chapter on numerical estimation and just reference it in a discussion like the above). Of course, the real answer comes from experimentation. I suspect Michael will be working on this with his new video stimuli. I'm actually not that interested in beating on the textual stimuli, as I strongly suspect they will have less uptake (and perhaps less effect) than, e.g., video.
 
\section{A Note on Demographics}

While a random sample of workers on Mechanical Turk yielded a relatively
Democratic and GW-accepting population, we were still able to observe
conservatives within this sample.

There may have been some self-selection at play here. But, given the applied
nature of this work, self-selection for this experiment likely reflects
self-selection for engagement with this material on-line. This would do little,
then, to erode the claim that on-line science and numeracy interventions can
have a net positive impact on individuals. Note also that the “polarization”
claim seems highly untenable in the face of results from
Chapter~\ref{chap:evilndi} (given how UC Berkeley undergraduates are from a
liberal pool of Americans).

One demographic oddity worth noting is our lack of self-identified “protestants”
from our samples recruited from Mechanical Turk. Certainly, one might expect
evangelicals would avoid materials such as those presented in the studies above.
But there remains a large, relatively liberal segment of the protestant
community. As it stands, this point remains a mystery, and might perhaps be the
subject of further study. However, I would again emphasize that applied research
such as that presented in this dissertation should focus on populations that one
is likely to engage with a real-world educational intervention!

 % regarding your demographic comments, I absolutely agree. Interestingly, according to my CEO at Oroeco, latinos and asians actually express greater concern for green issues than (american) whites. But I haven't had time to check into the literature here. Likewise, there's literature on the distribution of folks on MTurk (its similar to demographics for "the internet" in general, but I don't remember about the liberal skew - certainly it's higher SES than a truly representative U.S. sample).

\section{On Retrospective Reports of Surprise}

It is also interesting to note the different amounts of surprise reported for
two numbers in the interventions in Chapter~\ref{chap:prondi} versus the
surprise reported for the 400 words in Chapter~\ref{chap:mechanism}.
Specifically, while the 400 words \emph{contains} two of those numbers,
individuals report \emph{less} surprise to the textual mechanistic description.
An interesting analogy is provided by the account of retrospection on
colonoscopies in \textcite{kahneman_perspective_2003}. Here, participants report
retrospective pain that is a weighted combination of their peak experience of
pain, and the pain at the end of the procedure. Thus, while participants may
experience a blip of surprise in the 400 words, the fact that it ends with a
summary of the preceding information almost certainly means it ends on an
unsurprising note.  Thus, participants \emph{may} in fact experience comparable surprise
in both styles of intervention, but when surprise is reported only after the
unsurprising conclusion of the 400 words, this may drive retrospective surprise
down.

Less theoretically interesting, perhaps, but still practically useful, is that
we've seen above that different formulations of surprise may be more relevant to
different forms of information. In particular, participants appear more willing
to say that they were “surprised or embarrassed at their own lack of knowledge”
for our mechanistic explanation (as seen in the study in
Section~\ref{sec:mech-mturk}), while they were more willing to report straight
surprise at numerical information (as seen in the study in
Section~\ref{sec:pro-mturk}). Use of the appropriate question may allow for
better assessment of individual variability in a given intervention. Some care
should be exercised, as certain forms of question appear to be rated uniformly
high (or low) for a given participant (as seen in
Figure~\ref{fig:CCO-prondi-surp-corr}). Such differences might be hashed out in
the development of more refined assessments for future studies.


% \section{An Informal Comparison of Effect Sizes}

\section{Summary and Recommendations}

Is there value in educating individuals about climate change? In part, it
depends on who you are. If you are a science educator, I hardly needed to
complete this research to tell you that the answer is an emphatic \emph{yes}.
But importantly, by using surprising information that stays close to verifiable
facts, you may likely avoid problems with polarization. If you are seeking to
influence behavior or policy, however, it is a complex task.  But in contrast to
the view that climate education is dangerously polarizing, it seems that science
education might at least push us in the right direction (even if it, alone, is
not the most effective or efficient route to conservation). But in the end, if
it turns out that we are indeed educable in a meaningful way, there is perhaps
more value in conserving the environment that sustains us than if we were mere
automata to be shoved around with propaganda!

% Maybe put a reference to CRED / Elke Weber on finite pool of worry, and the
% kinds of problems we run into with Lomberg and even the Harte vs. Schrag axis.
% In the end, this stuff _must_ be informed by science. 

At the highest level, the question becomes, what would we have govern us as a
society at the highest level: Superstition, the preservation of national power,
market forces, or our best shot at objective truth? As we've seen in some of the
discussion above, these approaches are intertwined. But I hope I've made it
clear that, apart from philosophical arguments in it's favor, science education
seems to be an effective approach to tackling the behavioral problem of climate
change.
