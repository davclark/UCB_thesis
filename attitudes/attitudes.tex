\chapter{Survey methods}
\label{chap:survey}

Chapters~\ref{chap:mechanism}, \ref{chap:evil-ndi}, and \ref{chap:pro-ndi} all
utilize similar survey methods to assess climate-related beliefs and attitudes,
in addition to a number of related constructs relevant to Ranney's
\citeyear{ranney-rtmd} RTMD theory. Here we describe the nature of these
surveys, and our methods for analyzing them. For reference, the full list of
survey items used in this body of research is included in
Appendix~\ref{app:survey-items}.

\section{Clarifying “beliefs” and “attitudes”}

Survey methods in the social sciences may use the terms “belief” and “attitude”
in numerous ways. For example, an “attitude” may refer to a measured response,
whereas a “belief” may refer to a latent variable that explains a number of such
responses \cite{some-attitude-latent-var-ref}. Here, we take a more
common-language approach. Specifically, in the text that follows, a “belief” is
a measure of agreement with an objectively verifiable fact about the world (For
example. For example, the reality of anthropogenic climate change (item gw1_2)
may be difficult to ascertain, but in the end, it is something that could be
settled by observation. An “attitude,” on the other hand, is a measure of
agreement with an emotional stance towards some aspect of the world. For
example, worry about global warming (gw2_3).

\section{An overview of survey items}

Primarily taken from Callie’s master’s thesis. The first page in particular was
selected as the most targeted set of 6 questions targeting the 6 RTMD
constructs \cite{ranney-rtmd}.

\textbf{Include the RTMD graph here.}

\section{Na\"ive survey results}

Most of the climate-related interventions that follow include some measure of
participant attitudes and beliefs prior to the intervention. In this thesis, we
report on insights into different forms of \emph{intervention}. Thus, a detailed
consideration of these na\"ive
results is beyond our scope. However, some of the relationships obtained seem
relevant to understanding the mind of our potential students. I thus note such
results below. For a fuller treatment of survey material, please consult the
relevant publications of the Reasoning group
\cite[notably,][]{cohen-thesis,ranney-etal-2012}.

We start by noting that the number of potential relationships between the many
variables we have measured would require an enormous amount of data to test
fully. As such, we will restrict ourselves primarily to the exploration of
\emph{a priori} relationships of interest.

Include multi-correlation figures here.

Evo—GW link
Knowledge / self-knowledge (this isn’t really part of this section)
Other stuff? Maybe nationalism GW link? How does this tie generally into our
theoretical framework? How does it compare and contrast with Ranney’s previous
work?

