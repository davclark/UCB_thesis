\graphicspath{{pro-ndi/}}

\chapter{Learning Representative Climate-Relevant Numbers}
\label{chap:prondi}

Notes - contrast with mechanism study. Much less scaffolding, more burden on the
student to make sense, more burden on instructional design to make things clear.

\section{Actual Abstract (delete heading before submission)}

\section{Altering Beliefs with Factual Numbers}

This came from the CogSci 2013 paper - edit or cut

The Numerically-Driven Inferencing (NDI) paradigm has yielded marked
attitudinal and conceptual shifts with quite minimalist interventions. NDI and
one of its procedures, EPIC (both introduced by Ranney and colleagues in 2001),
represent a particularly compact, well-specified intervention. In EPIC,
participants (1) provide an Estimate for each policy-relevant item’s quantity;
(2) state a preferred target (or monetary allocation) Policy (or Preference) for
each quantity; (3) receive actual quantities as feedback to Incorporate (as new
``Information''); and (4) indicate whether one’s policy has Changed upon receiving
feedback. With even a single well-selected quantity, the EPIC procedure’s
feedback often shifts participants’ attitudes. Conceptual changes resulting from
EPIC are often remarkably durable for such a minimal intervention (e.g., Ranney
et al., 2008), as evidenced by increased estimation accuracy up to 12 weeks
after the procedure (Munnich, Ranney, \& Bachman, 2005). Therefore, we sought to
employ NDI interventions in addition to the mechanism intervention from Study 1
and before. Specifically, we presented different participant groups with
numerical information that is relevant to global climate change acceptance. We
used numbers that were likely to boost acceptance (Study 3), as well as numbers
that we thought might erode individual's belief in climate change (Study 2).

As before, survey methods employed in the following studies are described in
detail in Section~\ref{sec:survey}.

\section{Study 1: Online experiment with UC Undergrads}

Given the efficacy of “evil” numbers and previous successes of the NDI paradigm,
this study assessed the efficacy of numbers that support the claim of global
climate change. Again partly tongue-in-cheek, we call these “saintly” numbers.
Given prior NDI studies of similarly “shocking” magnitudes (e.g., Garcia de
Osuna, et al., 2004), our hypothesis was that the accurate feedback would
increase participants’ climate change acceptance, but diminish self-confidence
in their knowledge of the issue.


\subsection{Methods}

\subsubsection{Materials and Procedure}

This study used an on-line version of materials, as did Study 1, and used a
pre-test survey that was completed, on average, 18-days prior to the
intervention. In the main intervention, we queried individuals about eight
quantities (listed in Appendix~\ref{app:numbers}). The eight items were
accompanied by questions directed at participants’ surprise and their reactions
to each number. Fictitious monetary policies were left out of this version, as
simple attitude shifts were readily observed in the simplified 8-item “evil”
intervention, and these shifts are more directly comparable across experiments.
An added feature of the online intervention is that we could remind individuals
of the estimates they gave on the same page on which they incorporated numerical
feedback, ensuring that they contrasted the two. As with online surveys in
Chapter~\ref{chap:mechanism}, an attitude and belief post-test was administered
immediately after our intervention and also after a retention interval.
% TODO: What was retention interval?

\subsubsection{Participants}

UC Berkeley undergrads ($N=60$) were recruited via the Research Participation
Pool (RPP). The RPP pre-test was completed by 30 of these participants. 
% TODO: Demographics

\subsubsection{Analysis}

Analyses here were largely analagous to those described in
Section~\ref{sec:mech-online-methods}.

\subsection{Results}

 because
these items (as with the “evil items) were, as anticipated, able to
significantly erode self-rated knowledge (5.3 to 4.0, t(29)=-3.6, p<0.01). This
erosion was comparable to that found with the “evil” numbers. These items also
ranked relatively high on participant surprise compared to the 400-words from
Study 1. Mean surprise ratings across items was 4.8, while surprise ratings for
the 400 words was 2.9 (all ratings above “1” indicate some level of surprise).
% TODO: compare with surprise for evil numbers here

One of the most surprising numbers (at 5.2) was the percentage of active
researchers who support the tenets of anthropogenic climate change, reflective
of the strong relationship between perceived scientific consensus and acceptance
of climate change reported in Lewandowsky, Gignac, and Vaughan (2013). The two
numbers most comparable to the statistics in the 400 words were similarly
surprising, with the rises in atmospheric methane and atmospheric CO2 ranking at
5.9 and 5.1, respectively.


\subsubsection{GW attitutudes}

In spite of these powerful impacts attitudes, acceptance, and beliefs regarding
climate change remained stable after this intervention with “saintly” numbers
(6.71 pre and 6.67 post).  This lack of effect is counter to prior NDI studies
(as well as the results reported in Chapter~\ref{chap:mechanism}), in which
individuals’ preferences and beliefs were often markedly shifted by even a
single number. 

\subsection{Discussion}

Regarding surprise in comparison to mechanism, we can make a parallel here to
Kahneman's pain studies / retrospective vs experiencing.

Note that not all items had sources. Many were likely difficult to understand
(wordings in Appendix~\ref{app:numbers} are reflective of the final wordings
used in Study 2). Unlike Study 2, this intervention \emph{did} include the
assertion that the study involved no deceptions.

An experimental silver lining here is the demonstration that
participants will not report greater climate change acceptance merely by dint of
experimenter demand! One possible explanation is due to a methodological change:
In prior NDI and RTMD studies, participants were explicitly told that all
feedback statistics and other information were fully accurate (and that the
study involved no deceptions). This was \emph{also} the case in the experiment,
so such assertions are clearly not sufficient to drive changes in attitude and
belief.  One difference that \emph{may} partially account for our lack-of-effect
is that many previous studies also provided the particular scientific/literature
source both for each statistic that was sought and each provided as feedback.

So, it is possible that participants were less compelled by the authority of
this study’s statistics, compared to those in Chapter~\ref{chap:evilndi}.
Another possibility is that, as in Chapter~\ref{chap:evilndi}, participants were
left feeling less knowledgeable—weakening any boost these surprising numbers
could have on climate change acceptance.  

A final possiblity is that the effect of this numerical intervention would be
strengthened by an appropriate context for integrating this information. That
is, perhaps we could not simply present our numerical information as it was in
isolation with the expectation of an effect. Indeed, as we report in
\cite{clark_knowledge_inpress}, similar numbers had little immediate effect on
high-school students as a part of a global warming mechanism curriculum.
However, students exposed to numbers like those in this Study retained the
effects of the curriculum to a greater extent than students in a control group.

\section{Study 2: Online intervention on Mechanical Turk}

\section{Methods}

\subsubsection{Materials and Procedure}

After the difficulty obtaining shifts in GW attitudes and beliefs above, I
engaged in a thorough examination of the wording of the items (also discovering
that one item was off by an order of magnitude). This process was relatively
informal, and consisted of showing the items to various naïve individuals and
asking them if they had any difficulty understanding them.

Note that we did not include instructions to the effect that NO deceptions were
used.

\section{Results}


Also - in both of the above studies, the inclusion of information about
authority was scant (e.g., “journal article” instead of PNAS).

\subsection{Discussion}

As compared with Study 1, the primary change was an increase in the fluency of
materials. While we should be careful in making comparisons across populations,
the similarity in the effects reported in Chapter~\ref{chap:mechanism} provide
some evidence that similar interventions perform similarly across UC Berkeley
undergrads and Mechanical Turkers. Thus, it seems reasonable to recommend that
careful attention be given to materials like those used in this chapter. Items
should be tested for comprehensibility with naïve individuals prior to
attempting to use them in a belief or behavioral change intervention.

\section{Summary and Conclusions}

Despite the “failure” of Study 1 above, it affords us a number of insights.
Critically, we cannot simply throw a set of numbers at Americans and expect that
to impact their beliefs and attitudes. A real silver lining here is
support for the fact that shifts, when we do observe them, are \emph{not} driven
merely by experimenter demand.
