\chapter{The 400 Words}
\label{chap:400}

\begin{center}
\textbf{How does climate change (``global warming'') work?  The mechanism of the
greenhouse effect} \\\relax
% Here and below, I could be using something like mchem, but \textsubscript is
% included with fixltx2e.
[Or: ``Why do some gases concern scientists---like carbon dioxide
(CO\textsubscript{2})---but not others, like oxygen?'']
\end{center}

Scientists tell us that human activities are changing Earth's atmosphere and
increasing Earth's average temperature. What causes these climate changes?

First, let's understand Earth's ``normal'' temperature: When Earth absorbs
sunlight, which is mostly visible light, it heats up. Like the sun, Earth emits
energy---but because it is cooler than the sun, Earth emits lower-energy infrared
wavelengths. Greenhouse gases in the atmosphere (methane, carbon dioxide, etc.)
let visible light pass through, but absorb infrared light---causing the
atmosphere to heat up. The warmer atmosphere emits more infrared light, which
tends to be re-absorbed---perhaps many times---before the energy eventually
returns to space. The extra time this energy hangs around has helped keep Earth
warm enough to support life as we know it. (In contrast, the moon has no
atmosphere, and it is colder than Earth, on average.)

Since the industrial age began around the year 1750, atmospheric carbon dioxide
has increased by 40\% and methane has increased by 150\%. Such increases cause
\emph{extra} infrared light absorption, further heating Earth above its typical
temperature range (even as energy from the sun stays basically the same).  In
other words, energy that gets to Earth has an even harder time leaving it,
causing Earth's average temperature to increase---producing global climate
change. 

[In molecular detail, greenhouse gases absorb infrared light because their
molecules can vibrate to produce asymmetric distributions of electric charge,
which match the energy levels of various infrared wavelengths. In contrast,
non-greenhouse gases (such as oxygen and nitrogen---that is, O\textsubscript{2}
and N\textsubscript{2}) don't absorb infrared light, because they have symmetric
charge distributions even when vibrating.]

Summary: (a) Earth absorbs most of the sunlight it receives; (b) Earth then
emits the absorbed light's energy as infrared light; (c) greenhouse gases absorb
a lot of the infrared light before it can leave our atmosphere; (d) being
absorbed slows the rate at which energy escapes to space; and (e) the slower
passage of energy heats up the atmosphere, water, and ground. By increasing the
amount of greenhouse gases in the atmosphere, humans are increasing the
atmosphere's absorption of infrared light, thereby warming Earth and disrupting
global climate patterns.

\emph{Shorter summary}: Earth transforms sunlight's \underline{visible} light
energy into \underline{infrared} light energy, which leaves Earth slowly because
it is absorbed by greenhouse gases. When people produce greenhouse gases, energy
leaves Earth \underline{even more slowly}---\underline{raising} Earth's temperature.
