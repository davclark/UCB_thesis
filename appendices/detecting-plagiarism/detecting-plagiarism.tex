\chapter{Detecting Plagiarism using Google Custom Search Engine}
\label{app:detecting-plagiarism}

% Do I need to do some TM stuff here?

First, one needs to set up a custom search engine at
\url{http://www.google.com/cse}. A small number of queries are provided for free
each day, beyond which the service requires a small payment per 1,000 queries.
It's not obvious from the documentation, but to use the service one needs to
create a custom engine using some (any) URL. The logic is that this service is
for a “custom” search focused on your particular web properties. We, however,
simply wish to get generic google results for our texts. Thus, after creating
our custom-tailored search engine, in the settings, one can enable searching the
entire web, and then remove the initial URL.  Now, one has the ability to
search using Google's Custom Search API using one's developer key and custom
search engine ID. The results are probably identical with what a regular google
search would provide, but there is no guarantee from Google that this is the
case.

Google's API will accept a maximum of 32 search terms, and as such, I only used
the first 32 terms from each text for search. Each API call will return a number
of search hits, including “snippets” that match the text. In the case of
plagiarism, this will be an almost exact match. Thus, there is a drastic
decrease in “edit distance” in the case of copy-paste plagiarism. This can be
seen easily in Table~\ref{tab:plag-dists}.

\begin{table}
    \caption{Partial list of (sorted) distance measures from the top google hit for a
        number of texts. As you can see, the first three rows are markedly
        different from those that followed. The First column is a true
        “distance” and as such, larger numbers indicate less similarity. The
        “FuzzyWuzzy” measures are calculated as a percentage.}
    \label{tab:plag-dists}
    \centering
    \begin{tabular}{ccc}
        \toprule
        Levenshtein & FuzzyWuzzy & FuzzyWuzzy (partial) \\ 
        \midrule
        17	&  94	& 95 \\
        17	&  94	& 95 \\
        54	&  88	& 97 \\
        134	&  32	& 31 \\
        122	&  32	& 34 \\
        146	&  32	& 37 \\
        102	&  31	& 34 \\
        127	&  30	& 35 \\
        126	&  30	& 29 \\
        139	&  30	& 30 \\
        \bottomrule
    \end{tabular}
\end{table}

The difference appears to be coarse enough that a variety of metrics are
sufficient to separate out cases of plagiarism. Metrics evaluated were
“FuzzyWuzzy” (provided by SeatGeek at
\url{https://github.com/seatgeek/fuzzywuzzy}) and the classic Levenshtein, or
“edit” distance from the NLTK package (available at \url{http://nltk.org/}). Both
of these are measures of partial string similarity. The code used to perform
these analyses is available upon request.
