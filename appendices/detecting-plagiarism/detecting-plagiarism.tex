\chapter{Detecting Plagiarism using Google Custom Search Engine}
\label{app:detecting-plagiarism}

% Do I need to do some TM stuff here?

First, one needs to set up a custom search engine at
\href{http://www.google.com/cse}. It's not obvious from the documentation, but
one needs to create a custom engine using some (any) URL, which allows for
creation. After this, in the settings, one can enable searching the entire web,
as well as removing the initial URL. Now, one has the ability to search using
Google's Custom Search API using one's developer key and custom search engine
ID.

Google's API will accept a maximum of 32 search terms, and as such, I only used
the first 32 terms from each text for search. Each API call will return a number
of search hits, including “snippets” that match the text. In the case of
plagiarism, this will be an almost exact match. Thus, there is a drastic
decrease in “edit distance” in the case of copy-paste plagiarism. This can be
seen easily in the scree plot in Chapter~\ref{chap:mechanism}.

The difference appears to be coarse enough that a variety of metrics are
sufficient to separate out cases of plagiarism. Metrics evaluated were
“FuzzyWuzzy” and the classic Levenshtein, or “edit” distance. Code is available
upon request.
