\chapter{Using imputation to combine participants with and without a pre-test}
\label{app:imputation}

Imputation is a well-established approach to dealing with missing data (ref -
probably Keppel and Wickens). In a number of the experiments in this
dissertation, multiple groups received a similar intervention, but one group may
have been missing a pre-test where we obtained their na\"ive baseline score
(e.g., for a climate-relevant attitude). The approach we used in these cases was
to use the participants for which we \emph{did} have a pre-test score (i.e., our
sandwich group), and use the average of those as an approximation to our other
groups pre-test score. To be explicit, following is the exact R code used to
compute this test for Study 1 in Chapter~\ref{chap:mechanism}:

% If you have time and energy, it might be nice to type-set this better
% Also - looks like we can have about 78 characters like this, but that’s tight
\begin{minted}{r}
# Here, dfs is pre-populated with the measured values. We assign the mean of
# the sandwich (s) group scores to the pre-test scores for the no-pretest
# group (n). We then append the sandwich group scores unmodified.

imputed.df <- data.frame(pre.gw=mean(dfs$s.pre$total.gw),
                         total.gw=dfs$n.post$total.gw)

imputed.df <- rbind(imputed.df,
                    dfs$s.post[,c('pre.gw', 'total.gw')])

# Note - this gives the same result as a simple t-test on the difference
# scores, so we're not cheating on our degrees of freedom, or obtaining
# artificially lower variance on the pre-test scores.

with(imputed.df,
     t.test(pre.gw, total.gw, alternative='less', paired=TRUE) )
\end{minted}
