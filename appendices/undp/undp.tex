\chapter{UNDP Millenium Goals and Climate-Related Funding Choices}
\label{app:undp}

Chapter~\ref{chap:evilndi} describes a series of fund allocation policy
decisions made by participants. Below are the instructions given to participants
in our 2-item intervention, followed by the text used to describe the two
alternatives for each item.

\section{Funding Policy Instructions}

As a result of the UN Millennium Summit, in the year 2000, the United Nations
adopted eight goals for increasing the economic and social conditions of the
world’s poorest countries, called the Millennium Development Goals.  These goals
are to: (1) end poverty and hunger, (2) achieve universal primary education, (3)
promote gender equity, (4) reduce child mortality rates, (5) improve maternal
health, (6) combat HIV/AIDS and other diseases, (7) ensure environmental
sustainability, and (8) develop a global partnership for development. 

Imagine that you have been hired as a consultant to the United Nations. Your
task is to allocate funds between projects oriented toward global climate change
and projects focused on achieving other Millennium Development Goals. You will
provide from two to four policy allocations in total. 

For each policy, first you will be asked to estimate the value of a
policy-relevant statistic. Then you will make an initial policy recommendation.
You will be asked to describe your estimation process---in particular, what
knowledge and reasoning you used to make your estimate. (\emph{You will write
    all of this information inside of this packet.})

After making an initial recommendation for each of the Millennium Development
Goals, you will put this packet away, and begin on Packet 2 (the yellow packet).
At that time, you will be given the true values of the statistics, as well as an
opportunity to revise each recommendation you made.

\section{Funding Alternatives}

All 4 variants of the 2-item intervention used the same policy choices. The
first (policy one) was:
\begin{enumerate}
    \item Create initiatives to reduce extreme poverty and hunger; or
    \item Invest in new technologies to reduce the levels of greenhouse gases in the atmosphere.
\end{enumerate}

The second (policy two) was:
\begin{enumerate}
    \item Invest in providing sustainable access to safe drinking water and basic sanitation; or
    \item Invest in renewable energy technologies, such as solar and wind power.
\end{enumerate}
